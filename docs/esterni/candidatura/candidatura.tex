\documentclass[a4paper, 12pt]{article}

\newcommand{\templates} {../../template}
\usepackage[a4paper, margin=2.5cm]{geometry}
\usepackage{fancyvrb}
\usepackage{stackengine}
\usepackage{lastpage}

\usepackage{enumitem}
\setlist[itemize]{noitemsep}
\setlist[enumerate]{noitemsep}

\let\oldpar\paragraph
\renewcommand{\paragraph}[1]{\oldpar{#1\\}\noindent}

% Avoid dots in the table of contents, it mess with the gulpease calculation
\makeatletter
\renewcommand{\@dotsep}{10000} 
\makeatother

\newcommand{\makeindexdetails}{
	\pagestyle{fancy}
	\lhead{4ourSquared} \chead{\includegraphics[width=1cm]{../../template/4ourSquared_logo} } \rhead{Versione e Indice}
	\pagenumbering{Roman}
}

\newcommand{\makecontentsdetails}[1]{
	\clearpage
    \renewcommand{\footrulewidth}{0.4pt}
	\pagestyle{fancy}
	\lhead{4ourSquared} \chead{\includegraphics[width=1cm]{../../template/4ourSquared_logo}} \rhead{\nouppercase{\leftmark}}
	\pagenumbering{arabic}
    \lfoot{#1} \cfoot{} \rfoot{\thepage / \pageref{LastPage}}
}
\usepackage{graphicx}
\usepackage{hyperref}
\usepackage{makecell}

\newcommand{\settitolo}[1]{\newcommand{\titolo}{#1\\}}
\newcommand{\setprogetto}[1]{\newcommand{\progetto}{#1\\}}
\newcommand{\setcommittenti}[1]{\newcommand{\committenti}{#1\\}}
\newcommand{\setredattori}[1]{\newcommand{\redattori}{#1\\}}
\newcommand{\setrevisori}[1]{\newcommand{\revisori}{#1\\}}
\newcommand{\setresponsabili}[1]{\newcommand{\responsabili}{#1\\}}
\newcommand{\setversione}[1]{
	\ifdefined\versione\renewcommand{\versione}{#1\\}
	\else\newcommand{\versione}{#1\\}\fi
}
\newcommand{\setdestuso}[1]{\newcommand{\uso}{#1\\}}
\newcommand{\setdescrizione}[1]{\newcommand{\descrizione}{#1\\}}

\newcommand{\makefrontpage}{
	\begin{titlepage}
		\begin{center}

		\includegraphics[width=0.4\textwidth]{\templates/4ourSquared_logo}\\

		{\Large 4OURSQUARED}\\[6pt]
		\href{mailto://4oursquared.unipd@gmail.com}{4oursquared.unipd@gmail.com}\\
		
		\ifdefined\progetto
		\vspace{1cm}
		{\Large\progetto}
		{\large\committenti}
		\else\fi
		
		\vspace{1.5cm}
		{\LARGE\titolo}
		
		\vfill
		
		\begin{tabular}{r | l}
		\multicolumn{2}{c}{\textit{Informazioni}}\\
		\hline
		
		\ifdefined\redattori
			\textit{Redattori} &
			\makecell[l]{\redattori}\\
		\else\fi
		\ifdefined\revisori
			\textit{Revisori} &
			\makecell[l]{\revisori}\\
		\else\fi
		\ifdefined\responsabili
			\textit{Responsabili} &
			\makecell[l]{\responsabili}\\
		\else\fi
		
		\ifdefined\versione
			\textit{Versione} & \versione
		\else\fi
		
		\textit{Uso} & \uso
		
		\end{tabular}
		
		\vspace{2cm}
		
		\ifdefined\descrizione
		Descrizione
		\vspace{6pt}
		\hrule
		\descrizione
		\else\fi
		\end{center}
	\end{titlepage}
}
\usepackage{hyperref}
\usepackage{array}
\usepackage{tabularx}
\usepackage{adjustbox}

\newcounter{verscount}
\setcounter{verscount}{0}
\newcommand{\addversione}[5]{
	\ifdefined\setversione
		\setversione{#1}
	\else\fi
	\stepcounter{verscount}
	\expandafter\newcommand%
		\csname ver\theverscount \endcsname{#1&#2&#3&#4&#5}
}

\newcommand{\listversioni}{
	\ifnum\value{verscount}>1
		\csname ver\theverscount \endcsname
		\addtocounter{verscount}{-1}
		\\\hline
		\listversioni
	\else
		\csname ver\theverscount \endcsname\\\hline
	\fi
}

\newcommand{\makeversioni}{
	\begin{center}
		\begin{tabularx}{\textwidth}{|c|c|c|c|X|}
		\hline
		\textbf{Versione} & \textbf{Data} & \textbf{Redattore} & \textbf{Verificatore} & \textbf{Descrizione} \\
		\hline
		\listversioni
		\end{tabularx}
	\end{center}
	\clearpage
}
\usepackage{eurosym}

\settitolo{Candidatura al Capitolato C2}
\setredattori{\\ Soldà Matteo \\ }
\setrevisori{\\ Alberti Nicolas \\Brotto Romina \\ Cavaliere Erica \\ Salami Lorenzo \\}
\setdestuso{ esterno}
\setdescrizione{
    Candidatura al Capitolato C2
}

\begin{document}

\makefrontpage

\section*{Scelta}
Con la presente, il gruppo ha scelto di candidarsi per il capitolato \textit{Lumos Minima} di \textit{Imola Informatica} per i seguenti motivi:
\begin{itemize}
    \item Interesse personale di tutti i membri del gruppo nell'ambito dell'\textit{IoT};
    \item Disponibilità, competenza e cortesia da parte del proponente, sopratutto nel proporre seminari per l'approfondimento di argomenti legati al progetto;
    \item Possibilità, rispetto agli altri capitolati, di poter interagire con qualcosa che sia effettivamente tangibile;
    \item Adesione alle politiche \textit{green} promosse dal proponente;
    \item Buone possibilità di sviluppo futuro e implementazione del progetto nella vita di tutti i giorni;
    \item Grande libertà nella scelta delle tecnologie da utilizzare.
\end{itemize}

\section*{Preventivo}
\begin{tabular}{|c|c|c|c|}
    \hline
    \textbf{Ruolo} & \textbf{Costo orario (\euro)} & \textbf{Ore} & \textbf{Prezzo (\euro)}\\
    \hline
    Responsabile & 30 & 70 & 2100\\
    \hline
    Amministratore & 20 & 50 & 1000\\
    \hline
    Analista & 25 & 100 & 2500\\
    \hline
    Progettista & 25 & 130 & 3250\\
    \hline
    Programmatore & 15 & 130 & 1950\\
    \hline
    Verificatore & 15 & 90 & 1350\\
    \hline\hline
    \multicolumn{2}{|c|}{\textbf{Scadenza consegna}} & \textbf{Ore tot.} & \textbf{Preventivo finale (\euro)}\\
    \hline
    \multicolumn{2}{|c|}{15/09/2023} & 570 & 12150\\
    \hline
\end{tabular}\\[8pt]
Calcolando un lavoro individuale di 95 ore, per una media di circa 5 ore settimanali a testa, si lavora per 20 settimane, ovvero 5 mesi. Iniziando la prima settimana di aprile, si termina la prima settimana di settembre. Abbiamo considerato due settimane di lasco in caso di imprevisti.

\end{document}
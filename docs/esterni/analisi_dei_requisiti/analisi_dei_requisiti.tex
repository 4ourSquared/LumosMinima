\documentclass[a4paper, 12pt]{article}
\usepackage{eurosym}
\usepackage{pdflscape}
\usepackage{pgfgantt}
\usepackage{pgfplots}


\newcommand{\templates}{../../template}
\usepackage[a4paper, margin=2.5cm]{geometry}
\usepackage{fancyvrb}
\usepackage{stackengine}
\usepackage{lastpage}

\usepackage{enumitem}
\setlist[itemize]{noitemsep}
\setlist[enumerate]{noitemsep}

\let\oldpar\paragraph
\renewcommand{\paragraph}[1]{\oldpar{#1\\}\noindent}

% Avoid dots in the table of contents, it mess with the gulpease calculation
\makeatletter
\renewcommand{\@dotsep}{10000} 
\makeatother

\newcommand{\makeindexdetails}{
	\pagestyle{fancy}
	\lhead{4ourSquared} \chead{\includegraphics[width=1cm]{../../template/4ourSquared_logo} } \rhead{Versione e Indice}
	\pagenumbering{Roman}
}

\newcommand{\makecontentsdetails}[1]{
	\clearpage
    \renewcommand{\footrulewidth}{0.4pt}
	\pagestyle{fancy}
	\lhead{4ourSquared} \chead{\includegraphics[width=1cm]{../../template/4ourSquared_logo}} \rhead{\nouppercase{\leftmark}}
	\pagenumbering{arabic}
    \lfoot{#1} \cfoot{} \rfoot{\thepage / \pageref{LastPage}}
}
\usepackage{graphicx}
\usepackage{hyperref}
\usepackage{makecell}

\newcommand{\settitolo}[1]{\newcommand{\titolo}{#1\\}}
\newcommand{\setprogetto}[1]{\newcommand{\progetto}{#1\\}}
\newcommand{\setcommittenti}[1]{\newcommand{\committenti}{#1\\}}
\newcommand{\setredattori}[1]{\newcommand{\redattori}{#1\\}}
\newcommand{\setrevisori}[1]{\newcommand{\revisori}{#1\\}}
\newcommand{\setresponsabili}[1]{\newcommand{\responsabili}{#1\\}}
\newcommand{\setversione}[1]{
	\ifdefined\versione\renewcommand{\versione}{#1\\}
	\else\newcommand{\versione}{#1\\}\fi
}
\newcommand{\setdestuso}[1]{\newcommand{\uso}{#1\\}}
\newcommand{\setdescrizione}[1]{\newcommand{\descrizione}{#1\\}}

\newcommand{\makefrontpage}{
	\begin{titlepage}
		\begin{center}

		\includegraphics[width=0.4\textwidth]{\templates/4ourSquared_logo}\\

		{\Large 4OURSQUARED}\\[6pt]
		\href{mailto://4oursquared.unipd@gmail.com}{4oursquared.unipd@gmail.com}\\
		
		\ifdefined\progetto
		\vspace{1cm}
		{\Large\progetto}
		{\large\committenti}
		\else\fi
		
		\vspace{1.5cm}
		{\LARGE\titolo}
		
		\vfill
		
		\begin{tabular}{r | l}
		\multicolumn{2}{c}{\textit{Informazioni}}\\
		\hline
		
		\ifdefined\redattori
			\textit{Redattori} &
			\makecell[l]{\redattori}\\
		\else\fi
		\ifdefined\revisori
			\textit{Revisori} &
			\makecell[l]{\revisori}\\
		\else\fi
		\ifdefined\responsabili
			\textit{Responsabili} &
			\makecell[l]{\responsabili}\\
		\else\fi
		
		\ifdefined\versione
			\textit{Versione} & \versione
		\else\fi
		
		\textit{Uso} & \uso
		
		\end{tabular}
		
		\vspace{2cm}
		
		\ifdefined\descrizione
		Descrizione
		\vspace{6pt}
		\hrule
		\descrizione
		\else\fi
		\end{center}
	\end{titlepage}
}
\usepackage{hyperref}
\usepackage{array}
\usepackage{tabularx}
\usepackage{adjustbox}

\newcounter{verscount}
\setcounter{verscount}{0}
\newcommand{\addversione}[5]{
	\ifdefined\setversione
		\setversione{#1}
	\else\fi
	\stepcounter{verscount}
	\expandafter\newcommand%
		\csname ver\theverscount \endcsname{#1&#2&#3&#4&#5}
}

\newcommand{\listversioni}{
	\ifnum\value{verscount}>1
		\csname ver\theverscount \endcsname
		\addtocounter{verscount}{-1}
		\\\hline
		\listversioni
	\else
		\csname ver\theverscount \endcsname\\\hline
	\fi
}

\newcommand{\makeversioni}{
	\begin{center}
		\begin{tabularx}{\textwidth}{|c|c|c|c|X|}
		\hline
		\textbf{Versione} & \textbf{Data} & \textbf{Redattore} & \textbf{Verificatore} & \textbf{Descrizione} \\
		\hline
		\listversioni
		\end{tabularx}
	\end{center}
	\clearpage
}
\graphicspath{ {./immagini/} }

\settitolo{Analisi dei requisiti}
\setredattori{Erica Cavaliere \\ Francesco Ceccato}
\setdestuso{esterno}
\setdescrizione{
Questo documento si occupa di riportare un'analisi di tutti gli elementi richiesti e i vincoli da rispettare per una completa comprensione del progetto.
}


\addversione{0.0.1}{05/05/2023}{Erica Cavaliere}{Lorenzo Salami}{Stesura iniziale}
\addversione{0.0.2}{05/05/2023}{Francesco Ceccato}{Lorenzo Salami}{Inserimento di alcuni casi d'uso}

\def\pgfcalendarmonthitname#1{%
\ifcase#1 \or Gennaio\or Febbraio\or Marzo\or Aprile\or Maggio\or Giugno\or Luglio\or Agosto\or Settembre\or Ottobre\or Novembre\or Dicembre\fi%
}
\usepgfplotslibrary{dateplot}

\begin{document}
\makeindexdetails
\makefrontpage \makeversioni
\tableofcontents
\newpage
\clearpage
\makecontentsdetails{Analisi dei requisiti} 

\section{Introduzione}
In questo progetto viene richiesto di creare un sistema che permetta di gestire i lampioni, accendendo una o più luci se sono presenti nelle vicinanze una o più persone o spegnendole altrimenti.\newline
Bisognerà che ci sia anche un modo per registrare i guasti degli impianti e segnalarli tramite apposita applicazione.\newline

\newpage

\section{Casi d'uso}

\subsection{Diagramma dei casi d'uso}

\includegraphics[scale=0.7]{diagramma_use_case_1.png}

\includegraphics[scale=0.65]{diagramma_use_case_2.png}

\includegraphics[scale=0.7]{diagramma_use_case_3.png}

\includegraphics[scale=0.7]{diagramma_use_case_4.png}

\subsection{Attori}
\begin{itemize}
    \item Operatore non autenticato: operatore che non ha inserito le proprie credenziali
    \item Operatore autenticato: operatore che ha inserito correttamente le credenziali
    \item Sistema di autenticazione: sistema di controllo che permette di verificare il corretto inserimento dei dati per accedere al programma
    \item Sensore: dispositivo che rileva la presenza di persone
\end{itemize}

\subsection{Lista dei casi d'uso}
\subsubsection{UC1 - Rilevamento presenza di persone}
\textbf{Attore:} Sensore\newline
\textbf{PRE:} l'individuo non è ancora posizionato in prossimità dell'area con lampioni, i lampioni sono spenti\newline
\textbf{POST:} l'individuo è posizionato all'interno dell'area, i lampioni illuminano l'area\newline
\textbf{Scenario principale:}
\begin{enumerate}
    \item Il sensore rileva la presenza di uno o più individui nel raggio d'azione
    \item Il sistema riceve in modalità push/pull le informazioni dal sensore
    \item Il sistema elabora l'informazione ricevuta, aumenta l'intensità luminosa dell'area (include[UC17]) per un certo tempo
    \item A seguire, l'intensità luminosa dell'area viene riportata al valore di default
\end{enumerate}

\subsubsection{UC2 - Associazione di un nuovo sensore ad un'area illuminata}
\textbf{Attore:} operatore autenticato\newline
\textbf{PRE:} il sensore è fisicamente presente in un'area, ma non è configurato per essere parte del sistema gestito dall'applicazione\newline
\textbf{POST:} il sensore è inserito nel sistema ed è raggiungibile.\newline
\textbf{Scenario principale:}
\begin{enumerate}
    \item L'operatore accede al sistema
    \item L'operatore effettua il login con proprie credenziali
    \item L'operatore avvia procedura inserimento
    \item L'operatore specifica tipo di interazione push/pull, dettagli, posizione geografica dispositivo, raggio d'azione
    \item L'operatore specifica l'area illuminata di riferimento
    \item Viene ottenuta la conferma dell'inserimento
\end{enumerate}

\subsubsection{UC3 - Acquisizione intensità luminosa di un impianto}
\textbf{Attore:} operatore autenticato\newline
\textbf{PRE:} un'area luminosa configurata illumina con un intensità iniziale arbitraria\newline
\textbf{POST:} l'area luminosa summenzionata illumina con una precisa intensità finale\newline
\textbf{Scenario principale:}
\begin{enumerate}
    \item L'operatore accede al sistema
    \item L'operatore effettua il login con proprie credenziali
    \item Il sistema acquisisce l'elenco di tutte le aree illuminate
    \item L'operatore seleziona una o più aree illuminate
    \item L'operatore imposta un valore di intensità luminosa per tutti gli impianti nelle aree selezionate in 4.
    \item Il sistema configura tutti gli impianti selezionati in 4. all'intensità selezionata in 5.
\end{enumerate}

\subsubsection{UC4 - Aumento manuale dell'intensità luminosa di un'area illuminata}
\textbf{Attore:} operatore autenticato\newline
\textbf{PRE:} un'area luminosa configurata illumina con un intensità iniziale arbitraria\newline
\textbf{POST:} l'area luminosa summenzionata illumina con una precisa intensità finale\newline
\textbf{Scenario principale:}
\begin{enumerate}
    \item L'operatore accede al sistema
    \item L'operatore effettua il login con proprie credenziali
    \item Il sistema mostra elenco di tutte le aree illuminate
    \item L'operatore seleziona una o più aree illuminate
    \item L'operatore imposta un valore maggiore di intensità luminosa per tutti gli impianti nelle aree selezionate in 4.
    \item Il sistema configura tutti gli impianti selezionati in 4. all'intensità selezionata in 5.
\end{enumerate}

\subsubsection{UC5 - Diminuzione manuale dell'intensità luminosa di un'area illuminata}
\textbf{Attore:} operatore autenticato\newline
\textbf{PRE:} un'area luminosa configurata illumina con un intensità iniziale arbitraria\newline
\textbf{POST:} l'area luminosa summenzionata illumina con una precisa intensità finale\newline
\textbf{Scenario principale:}
\begin{enumerate}
    \item L'operatore accede al sistema
    \item L'operatore effettua il login con proprie credenziali
    \item Il sistema mostra elenco di tutte le aree illuminate
    \item L'operatore seleziona una o più aree illuminate
    \item L'operatore imposta un valore minore di intensità luminosa per tutti gli impianti nelle aree selezionate in 4.
    \item Il sistema configura tutti gli impianti selezionati in 4. all'intensità selezionata in 5.
\end{enumerate}

\subsubsection{UC6 - Login di un operatore}
\textbf{Attore:} operatore non autenticato\newline
\textbf{PRE:} l'operatore non è entrato nel sistema e quindi non può gestire i sistemi di illuminazione, ma è registrato nel database\newline
\textbf{POST:} l'operatore ha inserito le proprie credenziali e può gestire i sistemi di illuminazione\newline
\textbf{Scenario principale:}
\begin{enumerate}
    \item L'operatore accede al sistema
    \item L'operatore inserisce le proprie credenziali
    \item Il sistema verifica se le credenziali corrispondono a quelle di un utente nel database
\end{enumerate}
\textbf{Estensioni:}
\begin{itemize}
    \item [a.] Le credenziali inserite non sono corrette
    \begin{enumerate}
        \item Viene visualizzato un errore
        \item L'operatore deve immettere nuovamente le proprie credenziali
    \end{enumerate}
\end{itemize}

\subsubsection{UC7 - Logout dell'operatore}
\textbf{Attore:} operatore autenticato\newline
\textbf{PRE:} l'operatore ha il consenso di gestire i sistemi di illuminazione tramite l'applicazione\newline
\textbf{POST:} all'operatore non è consentito gestire i sistemi di illuminazione tramite l'applicazione, ma è registrato nel database\newline
\textbf{Scenario principale:}
\begin{enumerate}
    \item L'operatore ha l'accesso del sistema
    \item L'operatore seleziona il pulsante di Logout
    \item L'applicazione termina la sessione dell'operatore
\end{enumerate}

\subsubsection{UC8 - Inserimento di nuovo impianto luminoso}
\textbf{Attore:} operatore autenticato\newline
\textbf{PRE:} l'impianto è fisicamente presente ma non è registrato nel sistema\newline
\textbf{POST:} l'impianto è stato registrato correttamente e sarà possibile gestirlo tramite applicazione\newline
\textbf{Scenario principale:}
\begin{enumerate}
    \item L'operatore accede al sistema
    \item L'operatore effettua il login con proprie credenziali
    \item L'operatore avvia procedura inserimento
    \item L'operatore specifica il sensore che gestirà l'impianto e i relativi dettagli
    \item Viene ottenuta la conferma dell'inserimento
\end{enumerate}

\subsubsection{UC9 - Rimozione di un impianto luminoso esistente}
\textbf{Attore:} operatore autenticato\newline
\textbf{PRE:} l'impianto è registrato nel sistema\newline
\textbf{POST:} l'impianto è stato cancellato dal database e non sarà possibile gestirlo dall'applicazione\newline
\textbf{Scenario principale:}
\begin{enumerate}
    \item L'operatore accede al sistema
    \item L'operatore effettua il login con proprie credenziali
    \item L'operatore avvia procedura di rimozione
    \item Viene visualizzato l'elenco degli impianti luminosi esistenti
    \item L'operatore seleziona l'impianto da rimuovere dal sistema
    \item Viene chiesta la conferma di eliminazione
    \item L'operatore conferma l'operazione
    \item Viene rimosso l'impianto luminoso dal sistema
    \item Viene ottenuta la conferma di rimozione
\end{enumerate}
\textbf{Estensioni:}
\begin{itemize}
    \item [a.] L'operatore non conferma la rimozione alla richiesta di conferma
    \begin{enumerate}
        \item Viene visualizzato l'elenco degli impianti luminosi esistenti
        \item L'operatore devrà selezionare un impianto da eliminare o annullare la procedura
    \end{enumerate}
    \item [b.] Viene annullata la procedura di rimozione
    \begin{enumerate}
        \item La lista degli impianti non subisce modifiche
        \item Viene visualizzata la schermata principale
    \end{enumerate}
\end{itemize}

\subsubsection{UC10 - Consultazione elenco impianti luminosi}
\textbf{Attore:} operatore autenticato\newline
\textbf{PRE:} l'operatore non visualizza l'elenco degli impianti luminosi\newline
\textbf{POST:} l'operatore visualizza l'elenco degli impianti e potrà interagire con esso\newline
\textbf{Scenario principale:}
\begin{enumerate}
    \item L'operatore accede al sistema
    \item L'operatore effettua il login con proprie credenziali
    \item L'operatore seleziona il pulsante di consultazione elenco impianti luminosi
    \item Viene visualizzato l'elenco degli impainti luminosi
\end{enumerate}

\subsubsection{UC11 - Consultazione elenco impianti guasti}
\textbf{Attore:} operatore autenticato\newline
\textbf{PRE:} l'operatore non visualizza l'elenco degli impianti con segnalato dei guasti\newline
\textbf{POST:} l'operatore consulta l'elenco degli impianti dei guasti e interagisce con esso\newline
\textbf{Scenario principale:}
\begin{enumerate}
    \item L'operatore accede al sistema
    \item L'operatore effettua il login con proprie credenziali
    \item L'operatore seleziona il pulsante di consultazione elenco impianti guasti
    \item Viene visualizzato l'elenco degli impainti guasti
\end{enumerate}

\subsubsection{UC12 - Inserimento manuale di un guasto ad un impianto}
\textbf{Attore:} operatore registrato\newline
\textbf{PRE:} è presente un impianto non funzionante che non è incluso nella lista degli impianti guasti\newline
\textbf{POST:} l'impianto non funzionante è incluso nella lista degli impianti guasti\newline
\textbf{Scenario principale:}
\begin{enumerate}
    \item L'operatore accede al sistema
    \item L'operatore effettua il login con proprie credenziali
    \item L'operatore avvia la procedura per l’inserimento di un impianto luminoso guasto
    \item Viene consultato l’elenco degli impianti di illuminazione guasti(include [UC10]).
    \item L'operatore marca il dispositivo interessato come guasto, scatenandone l’inserimento dell’elenco degli impianti di illuminazione guasti
\end{enumerate}

\subsubsection{UC13 - Consultazione elenco aree illuminate}
\textbf{Attore:} operatore autenticato\newline
\textbf{PRE:} l'operatore non visualizza l'elenco delle aree illuminate\newline
\textbf{POST:} l'operatore visualizza l'elenco delle aree illuminate e potrà interagire con esso\newline
\textbf{Scenario principale:}
\begin{enumerate}
    \item L'operatore accede al sistema
    \item L'operatore effettua il login con proprie credenziali
    \item L'operatore seleziona il pulsante di consultazione elenco aree illuminate
    \item Viene visualizzato l'elenco delle aree illuminate
\end{enumerate}

\subsubsection{UC14 - Creazione di nuova area illuminata}
\textbf{Attore:} operatore autenticato\newline
\textbf{PRE:} l'area illuminata non è presente nel sistema\newline
\textbf{POST:} l'area illuminata è presente nel sistema ed è possibile gestirla tramite applicazione\newline
\textbf{Scenario principale:}
\begin{enumerate}
    \item L'operatore accede al sistema
    \item L'operatore effettua il login con proprie credenziali
    \item L'operatore avvia la procedura di creazione di una nuova area illuminata
    \item L'operatore specifica posizione geografica e relativi dettagli
    \item Viene ottenuta la conferma di inserimento
\end{enumerate}

\subsubsection{UC15 - Modifica di area illuminata esistente}
\textbf{Attore:} operatore autenticato\newline
\textbf{PRE:} l'area illuminata è registrata con dati arbitrari\newline
\textbf{POST:} l'area illuminata è registrata con i dati aggiornati\newline
\textbf{Scenario principale:}
\begin{enumerate}
    \item L'operatore accede al sistema
    \item L'operatore effettua il login con proprie credenziali
    \item L'operatore avvia la procedura di modifica di un'area illuminata
    \item Viene visualizzato l'elenco delle aree illuminate
    \item L'operatore selezione l'area illuminata che desidera modificare
    \item Viene visualizzata la schermata di modifica dell'area illuminata selezionata in 5.
    \item L'operatore modifica l'area illuminata con dati aggiornati
    \item Viene ottenuta la conferma di modifica 
\end{enumerate}

\subsubsection{UC16 - Rimozione di area illuminata esistente}
\textbf{Attore:} operatore autenticato\newline
\textbf{PRE:} l'area illuminata è presente nel sistema e visibile tramite applicazione\newline
\textbf{POST:} l'area illuminata non è presente nel sistema\newline
\textbf{Scenario principale:}
\begin{enumerate}
    \item L'operatore accede al sistema
    \item L'operatore effettua il login con proprie credenziali
    \item L'operatore avvia la procedura di rimozione di un'area illuminata
    \item Viene visualizzato l'elenco delle aree illuminate esistenti
    \item L'operatore selezione l'area illuminata che desidera rimuovere
    \item Viene visualizzata la richiesta di conferma di cancellazione
    \item L'operatore conferma l'operazione
    \item Viene rimossa l'area illuminata dal sistema
    \item Viene ottenuta la conferma di rimozione
\end{enumerate}
\textbf{Estensioni:}
\begin{itemize}
    \item [a.] L'operatore non conferma la rimozione alla richiesta di conferma
    \begin{enumerate}
        \item Viene visualizzato l'elenco delle aree illuminate esistenti
        \item L'operatore dovrà selezionare un'area illuminata da eliminare o annullare la procedura
    \end{enumerate}
    \item [b.] Viene annullata la procedura di rimozione
    \begin{enumerate}
        \item La lista delle aree illuminate non subisce modifiche
        \item Viene visualizzata la schermata principale
    \end{enumerate}
\end{itemize}

\subsubsection{UC17 - Aumento automatico dell'intensità luminosa di un'area illuminata}
\textbf{Attore:} sensore\newline
\textbf{PRE:} un'area luminosa configurata illumina con un'intensità iniziale arbitraria\newline
\textbf{POST:} l'area luminosa summenzionata illumina con una precisa intensità finale\newline
\textbf{Scenario principale:}
\begin{enumerate}
    \item Il sensore rileva la presenza di persone in una area illuminata precisa [UC1]
    \item Il sensore aumenta l'intensità luminosa dell'area illuminata rilevata in 1.
\end{enumerate}

\subsubsection{UC18 - Diminuzione automatica dell'intensità luminosa di un'area illuminata}
\textbf{Attore:} sensore\newline
\textbf{PRE:} un'area luminosa configurata illumina con un'intensità iniziale arbitraria\newline
\textbf{POST:} l'area luminosa summenzionata illumina con una precisa intensità finale\newline
\textbf{Scenario principale:}
\begin{enumerate}
    \item Il sensore rileva che in un'area illuminata con intensità luminosa alta non ci sono persone presenti.
    \item Il sensore diminuisce l'intensità luminosa dell'area illuminata rilevata in 1.
\end{enumerate}

\subsubsection{UC19 - Settaggio di un impianto in modalità automatica}
\textbf{Attore:} operatore autenticato\newline
\textbf{PRE:} l'impianto non è stato settato con modalità automatica\newline
\textbf{POST:} l'impianto ha la modalità automatica attiva e può gestire i dispositivi collegati ad esso\newline
\textbf{Scenario principale:}
\begin{enumerate}
    \item L'operatore accede al sistema
    \item L'operatore effettua il login con proprie credenziali
    \item L'operatore avvia la procedura di settaggio di un impianto in modalità automatica
    \item L'operatore visualizza l'elenco degli impianti esistenti
    \item L'operatore seleziona l'impianto che desidera impostare con modalità automatica
    \item Viene ottenuta la conferma di attivazione della modalità automatica dell'impianto selezionato in 5.
\end{enumerate}

\subsubsection{UC20 - Rimozione di impianto da elenco impianti guasti}
\textbf{Attore:} operatore autenticato\newline
\textbf{PRE:} l'impianto è presente nel sistema come impianto guasto\newline
\textbf{POST:} l'impianto è presente nel sistema ma non è indicato come impianto guasto\newline
\textbf{Scenario principale:}
\begin{enumerate}
    \item L'operatore accede al sistema
    \item L'operatore effettua il login con proprie credenziali
    \item L'operatore avvia la procedura di rimozione di un impianto guasto
    \item Viene visualizzato l'elenco degli impianti guasti esistenti
    \item L'operatore seleziona l'impianto che desidera rimuovere dall'elenco
    \item Viene ottenuta la conferma di rimozione
\end{enumerate}

\end{document}
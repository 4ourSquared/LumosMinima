\documentclass[a4paper, 12pt]{article}
\usepackage{eurosym}
\usepackage{pdflscape}
\usepackage{pgfgantt}
\usepackage{pgfplots}
\usepackage{booktabs}
\usepackage{longtable}


\newcommand{\templates}{../../template}
\usepackage[a4paper, margin=2.5cm]{geometry}
\usepackage{fancyvrb}
\usepackage{stackengine}
\usepackage{lastpage}

\usepackage{enumitem}
\setlist[itemize]{noitemsep}
\setlist[enumerate]{noitemsep}

\let\oldpar\paragraph
\renewcommand{\paragraph}[1]{\oldpar{#1\\}\noindent}

% Avoid dots in the table of contents, it mess with the gulpease calculation
\makeatletter
\renewcommand{\@dotsep}{10000} 
\makeatother

\newcommand{\makeindexdetails}{
	\pagestyle{fancy}
	\lhead{4ourSquared} \chead{\includegraphics[width=1cm]{../../template/4ourSquared_logo} } \rhead{Versione e Indice}
	\pagenumbering{Roman}
}

\newcommand{\makecontentsdetails}[1]{
	\clearpage
    \renewcommand{\footrulewidth}{0.4pt}
	\pagestyle{fancy}
	\lhead{4ourSquared} \chead{\includegraphics[width=1cm]{../../template/4ourSquared_logo}} \rhead{\nouppercase{\leftmark}}
	\pagenumbering{arabic}
    \lfoot{#1} \cfoot{} \rfoot{\thepage / \pageref{LastPage}}
}
\usepackage{graphicx}
\usepackage{hyperref}
\usepackage{makecell}

\newcommand{\settitolo}[1]{\newcommand{\titolo}{#1\\}}
\newcommand{\setprogetto}[1]{\newcommand{\progetto}{#1\\}}
\newcommand{\setcommittenti}[1]{\newcommand{\committenti}{#1\\}}
\newcommand{\setredattori}[1]{\newcommand{\redattori}{#1\\}}
\newcommand{\setrevisori}[1]{\newcommand{\revisori}{#1\\}}
\newcommand{\setresponsabili}[1]{\newcommand{\responsabili}{#1\\}}
\newcommand{\setversione}[1]{
	\ifdefined\versione\renewcommand{\versione}{#1\\}
	\else\newcommand{\versione}{#1\\}\fi
}
\newcommand{\setdestuso}[1]{\newcommand{\uso}{#1\\}}
\newcommand{\setdescrizione}[1]{\newcommand{\descrizione}{#1\\}}

\newcommand{\makefrontpage}{
	\begin{titlepage}
		\begin{center}

		\includegraphics[width=0.4\textwidth]{\templates/4ourSquared_logo}\\

		{\Large 4OURSQUARED}\\[6pt]
		\href{mailto://4oursquared.unipd@gmail.com}{4oursquared.unipd@gmail.com}\\
		
		\ifdefined\progetto
		\vspace{1cm}
		{\Large\progetto}
		{\large\committenti}
		\else\fi
		
		\vspace{1.5cm}
		{\LARGE\titolo}
		
		\vfill
		
		\begin{tabular}{r | l}
		\multicolumn{2}{c}{\textit{Informazioni}}\\
		\hline
		
		\ifdefined\redattori
			\textit{Redattori} &
			\makecell[l]{\redattori}\\
		\else\fi
		\ifdefined\revisori
			\textit{Revisori} &
			\makecell[l]{\revisori}\\
		\else\fi
		\ifdefined\responsabili
			\textit{Responsabili} &
			\makecell[l]{\responsabili}\\
		\else\fi
		
		\ifdefined\versione
			\textit{Versione} & \versione
		\else\fi
		
		\textit{Uso} & \uso
		
		\end{tabular}
		
		\vspace{2cm}
		
		\ifdefined\descrizione
		Descrizione
		\vspace{6pt}
		\hrule
		\descrizione
		\else\fi
		\end{center}
	\end{titlepage}
}
\usepackage{hyperref}
\usepackage{array}
\usepackage{tabularx}
\usepackage{adjustbox}

\newcounter{verscount}
\setcounter{verscount}{0}
\newcommand{\addversione}[5]{
	\ifdefined\setversione
		\setversione{#1}
	\else\fi
	\stepcounter{verscount}
	\expandafter\newcommand%
		\csname ver\theverscount \endcsname{#1&#2&#3&#4&#5}
}

\newcommand{\listversioni}{
	\ifnum\value{verscount}>1
		\csname ver\theverscount \endcsname
		\addtocounter{verscount}{-1}
		\\\hline
		\listversioni
	\else
		\csname ver\theverscount \endcsname\\\hline
	\fi
}

\newcommand{\makeversioni}{
	\begin{center}
		\begin{tabularx}{\textwidth}{|c|c|c|c|X|}
		\hline
		\textbf{Versione} & \textbf{Data} & \textbf{Redattore} & \textbf{Verificatore} & \textbf{Descrizione} \\
		\hline
		\listversioni
		\end{tabularx}
	\end{center}
	\clearpage
}
\graphicspath{ {./immagini/} }

\settitolo{Analisi dei requisiti}
\setredattori{Nicolas Alberti \\ Romina Brotto \\ Erica Cavaliere \\ Francesco Ceccato }
\setdestuso{esterno}
\setdescrizione{
Questo documento si occupa di riportare un'analisi di tutti gli elementi richiesti e i vincoli da rispettare per una completa comprensione del progetto.
}


\addversione{0.0.1}{05/05/2023}{Erica Cavaliere}{Lorenzo Salami}{Stesura iniziale}
\addversione{0.0.2}{05/05/2023}{Francesco Ceccato}{Lorenzo Salami}{Inserimento di alcuni casi d'uso}
\addversione{0.0.3}{09/05/2023}{Brotto Romina}{Soldà Matteo}{Aggiunta sezioni 2, 2.1, 4, 4.1 ed inizio stesura requisiti funzionali}
\addversione{0.0.4}{10/05/2023}{Alberti Nicolas}{Brotto Romina}{}

\def\pgfcalendarmonthitname#1{%
\ifcase#1 \or Gennaio\or Febbraio\or Marzo\or Aprile\or Maggio\or Giugno\or Luglio\or Agosto\or Settembre\or Ottobre\or Novembre\or Dicembre\fi%
}
\usepgfplotslibrary{dateplot}

\begin{document}
\makeindexdetails
\makefrontpage \makeversioni
\tableofcontents
\newpage
\clearpage
\makecontentsdetails{Analisi dei requisiti} 

\section{Introduzione}
\subsection{Scopo del Documento}
In questo progetto viene richiesto di creare un sistema che permetta di gestire i lampioni, accendendo una o più luci se sono presenti nelle vicinanze una o più persone o spegnendole altrimenti.\newline
Bisognerà che ci sia anche un modo per registrare i guasti degli impianti e segnalarli tramite apposita applicazione.

\subsection{Riferimenti}
\subsubsection*{Riferimenti normativi}
\begin{itemize}
    \item Capitolato d'appalto: C2; 
    \item Norme di Progetto.
\end{itemize}

\subsubsection*{Riferimenti informativi}
\begin{itemize}
    \item Slide analisi dei requisiti - Materiale didattico del corso IS;
    \item Slide diagrammi dei casi d'uso - Materiale didattico del corso IS.
\end{itemize}
\newpage
\section{Descrizione del Prodotto}
L'azienda \textit{Imola Informatica} propone attraverso il capitolato C2:
\textit{Lumos Minima}. L'obiettivo è sviluppare un sistema per l'ottimizzazione
dell'illuminazione pubblica che permetta ai gestori di sfruttare la possibilità
di regolare l'intensità di luce emessa dagli impianti, grazie all'utilizzo di
sensori specifici che permettono di ottenere informazioni legate all'ambiente circostante.
\subsection{Scopo del Prodotto}
Il sistema sopra citato consentirebbe di garantire sicurezza stradale e sociale, e al tempo stesso permetterebbe di risparmiare energia e quindi risorse economiche ed ambientali. Il processo è caratterizzato da operazioni effettuate dal sistema e/o dai gestori:
\begin{itemize}
    \item Rilevamento della presenza di persone in prossimità della fonte luminosa;
    \item Aumento e riduzione dell'intensità luminosa;
    \item Rilevamento automatico del guasto di un impianto di illuminazione;
    \item Segnalazione manuale del guasto di un impianto di illuminazione;
    \item Aumento e riduzione manuale dell'intensità luminosa;
    \item Inserimento e gestione di un impianto luminoso;
    \item Aumento o riduzione globale dell'intensità luminosa. 
\end{itemize}

\subsection{Parti del Prodotto}
Il prodotto si compone delle seguenti parti: % BOZZA
\begin{itemize}
    \item Landing page per permettere l'autenticazione dell'operatore;
    \item Web App con dashboard per visualizzare, selezionare tutti i gruppi di
    impianti luminosi ed interagire con essi.
\end{itemize}
% BOZZA - DA DEFINIRE
Per ogni impianto luminoso deve essere prevista una modalità a funzionamento
automatico ed una modalità a funzionamento manuale, in cui l'operatore potrà
configurare a proprio piacimento gli elementi dell'impianto. 
\subsection{Caratteristiche utenti}
La Web App prevede due tipologie di utenti:
\begin{itemize}
    \item operatore non autenticato, che può: \begin{itemize}
        \item visualizzare la landing page;
        \item accedere al servizio previo possesso di credenziali autenticate.
    \end{itemize}
    \item operatore autenticato, che può: \begin{itemize}
        \item visualizzare tutti gli impianti luminosi gestiti dall'organizzazione;
        \item interagire con ogni impianto e modificarne il funzionamento;
        \item visualizzare eventuali errori e/o guasti.
    \end{itemize}
\end{itemize}
% BOZZA - DA DEFINIRE
Il prodotto si rivolge a tutte le organizzazioni che necessitano di gestire un
numero consistente di impianti luminosi, a loro volta composti da più elementi
quali luci e sensori. L'utente finale deve conoscere il funzionamento di tali
componenti, al fine di poter gestire nella maniera più adeguata gli impianti ed
inoltre deve saper interpretare gli errori forniti dal prodotto, per poter
correggere il funzionamento dell'impianto.

\subsection{Vincoli e preferenze}
Il proponente non impone particolari vincoli nella scelta delle tecnologie e dei linguaggi, sono stati però forniti alcui suggerimenti da prendere in considerazione:
\begin{itemize}
    \item utilizzo di React per lo sviluppo delle parti di Front-end;
    \item utilizzo di Node JS per lo sviluppo delle parti di Back-end;
\end{itemize}

Per il completamento del progetto il proponente richiede che siano ottenuti i
seguienti risultati:
\begin{itemize}
    \item applicazione Web Responsive che soddisfi i requisiti obblgatori
    illustrati dai casi d'uso;
    \item test che dimostrino il corretto funzionamento dei servizi e delle
    funzionalità previste, con una copertura minima dell'80\%;
    \item documentazione sulle scelte implementative e progettuali con le
    motivazioni e i problemi aperti ed eventuali soluzioni da esplorare.
\end{itemize}
Sono di interesse altri due risultati desiderabili ma non vincolanti al fine del
completamento del progetto:
\begin{itemize}
    \item cifratura di tutte le comunicazoni fra App e Server per garantire la
    validità delle informazioni;
    \item analisi riguardante sia il carico massimo supportato in numero di
    dispositivi che del servizio cloud più adatto per supportarlo
    analizzando prezzo, stabilità, del servizio ed assistenza.
\end{itemize} 
\newpage
\section{Casi d'uso}

\subsection{Diagramma dei casi d'uso}

\includegraphics[scale=0.7]{diagramma_use_case_1.png}

\includegraphics[scale=0.60]{diagramma_use_case_2.png}

\includegraphics[scale=0.7]{diagramma_use_case_3.png}

\includegraphics[scale=0.7]{diagramma_use_case_4.png}

\subsection{Attori}
\begin{itemize}
    \item Utente non autenticato: utente che non ha inserito le proprie credenziali;
    \item Utente autenticato: utente che ha inserito correttamente le credenziali;
    \item Sistema di autenticazione: sistema di controllo che permette di verificare il corretto inserimento dei dati per accedere al programma;
    \item Manutentore: operatore registrato che si occupa di configurare le aree luminose e sistemare eventuali guasti
    \item Amministratore: operatore registrato che oltre a poter configurare nuove aree, può aggiungere o eliminare sensori o impianti luminosi
    \item Sensore: dispositivo che si occupa di rilevare le persone e modificare l'intensità luminosa degli impianti di illuminazione
\end{itemize}

\subsection{Lista dei casi d'uso}
\subsubsection{UC1 - Rilevamento presenza di persone}
\textbf{Attore:} Sensore.\newline
\textbf{PRE:} l'individuo non è ancora posizionato in prossimità dell'area con lampioni, i lampioni sono spenti.\newline
\textbf{POST:} l'individuo è posizionato all'interno dell'area, i lampioni illuminano l'area.\newline
\textbf{Scenario principale:}
\begin{enumerate}
    \item Il sensore rileva la presenza di uno o più individui nel raggio d'azione;
    \item Il sistema riceve in modalità push/pull le informazioni dal sensore;
    \item Il sistema elabora l'informazione ricevuta, aumenta l'intensità luminosa dell'area (include[UC17]) per un certo tempo;
    \item A seguire, l'intensità luminosa dell'area viene riportata al valore di default.
\end{enumerate}
\textbf{Requisiti collegati:} RF1-O\newline

\subsubsection{UC2 - Acquisizione intensità luminosa di un'area illuminata}
\textbf{Attore:} Manutentore.\newline
\textbf{PRE:} un'area luminosa configurata illumina con un intensità iniziale arbitraria.\newline
\textbf{POST:} l'area luminosa summenzionata illumina con una precisa intensità finale.\newline
\textbf{Scenario principale:}
\begin{enumerate}
    \item L'operatore accede al sistema;
    \item L'operatore effettua il login con proprie credenziali;
    \item Il sistema acquisisce l'elenco di tutte le aree illuminate;
    \item L'operatore seleziona una o più aree illuminate;
    \item L'operatore imposta un valore di intensità luminosa per tutti gli impianti nelle aree selezionate in 4;
    \item Il sistema configura tutti gli impianti selezionati in 4. all'intensità selezionata in 5.
\end{enumerate}
\textbf{Requisiti collegati:} RF2-O\newline

\subsubsection{UC3 - Aumento manuale dell'intensità luminosa di un'area illuminata}
\textbf{Attore:} Manutentore.\newline
\textbf{PRE:} un'area luminosa configurata illumina con un intensità iniziale arbitraria.\newline
\textbf{POST:} l'area luminosa summenzionata illumina con una precisa intensità finale.\newline
\textbf{Scenario principale:}
\begin{enumerate}
    \item L'operatore accede al sistema;
    \item L'operatore effettua il login con proprie credenziali;
    \item Il sistema mostra l'elenco di tutte le aree illuminate;
    \item L'operatore seleziona una o più aree illuminate;
    \item L'operatore imposta un valore maggiore di intensità luminosa per tutti gli impianti nelle aree selezionate in 4;
    \item Il sistema configura tutti gli impianti selezionati in 4. all'intensità selezionata in 5.
\end{enumerate}
\textbf{Requisiti collegati:} RF3-O\newline

\subsubsection{UC4 - Diminuzione manuale dell'intensità luminosa di un'area illuminata}
\textbf{Attore:} Manutentore.\newline
\textbf{PRE:} un'area luminosa configurata illumina con un intensità iniziale arbitraria.\newline
\textbf{POST:} l'area luminosa summenzionata illumina con una precisa intensità finale.\newline
\textbf{Scenario principale:}
\begin{enumerate}
    \item L'operatore accede al sistema;
    \item L'operatore effettua il login con proprie credenziali;
    \item Il sistema mostra elenco di tutte le aree illuminate;
    \item L'operatore seleziona una o più aree illuminate;
    \item L'operatore imposta un valore minore di intensità luminosa per tutti gli impianti nelle aree selezionate in 4;
    \item Il sistema configura tutti gli impianti selezionati in 4. all'intensità selezionata in 5.
\end{enumerate}
\textbf{Requisiti collegati:} RF4-O\newline

\subsubsection{UC5 - Login}
\textbf{Attore:} 
\begin{itemize}
    \item Utente non atenticato;
    \item Sistema di autenticazione.
\end{itemize}
\textbf{PRE:} l'utente non è entrato nel sistema e quindi non può gestire i sistemi di illuminazione, ma è registrato nel database.\newline
\textbf{POST:} l'utente ha inserito le proprie credenziali e può gestire i sistemi di illuminazione.\newline
\textbf{Scenario principale:}
\begin{enumerate}
    \item L'utente accede al sistema;
    \item L'utente inserisce le proprie credenziali;
    \item Il sistema verifica se le credenziali corrispondono a quelle di un utente nel database;
\end{enumerate}
\textbf{Estensioni:}
\begin{itemize}
    \item [a.] Le credenziali inserite non sono corrette;
    \begin{enumerate}
        \item Viene visualizzato un errore;
        \item L'utente deve immettere nuovamente le proprie credenziali.
    \end{enumerate}
\end{itemize}
\textbf{Requisiti collegati:} RF5-O\newline

\subsubsection{UC6 - Logout}
\textbf{Attore:} Utente autenticato.\newline
\textbf{PRE:} l'utente ha il consenso di gestire i sistemi di illuminazione tramite l'applicazione.\newline
\textbf{POST:} all'utente non è consentito gestire i sistemi di illuminazione tramite l'applicazione, ma è registrato nel database.\newline
\textbf{Scenario principale:}
\begin{enumerate}
    \item L'utente ha l'accesso del sistema;
    \item L'utente seleziona il pulsante di Logout;
    \item L'applicazione termina la sessione dell'utente;
\end{enumerate}
\textbf{Requisiti collegati:} RF6-O\newline

\subsubsection{UC7 - Consultazione elenco aree illuminate}
\textbf{Attore:} Utente autenticato.\newline
\textbf{PRE:} l'utente non visualizza l'elenco degli impianti luminosi.\newline
\textbf{POST:} l'utente visualizza l'elenco degli impianti e potrà interagire con esso.\newline
\textbf{Scenario principale:}
\begin{enumerate}
    \item L'utente accede al sistema;
    \item L'utente effettua il login con proprie credenziali;
    \item L'utente seleziona il pulsante di consultazione elenco impianti luminosi;
    \item Viene visualizzato l'elenco degli impainti luminosi.
\end{enumerate}
\textbf{Requisiti collegati:} RF7-O\newline

\subsubsection{UC8 - Consultazione elenco aree illuminate con guasti}
\textbf{Attore:} Manutentore.\newline
\textbf{PRE:} il manutentore non visualizza l'elenco degli impianti con segnalato dei guasti.\newline
\textbf{POST:} il manutentore consulta l'elenco degli impianti dei guasti e interagisce con esso.\newline
\textbf{Scenario principale:}
\begin{enumerate}
    \item il manutentore accede al sistema;
    \item il manutentore effettua il login con proprie credenziali;
    \item il manutentore seleziona il pulsante di consultazione elenco impianti guasti;
    \item Viene visualizzato l'elenco degli impianti guasti.
\end{enumerate}
\textbf{Requisiti collegati:} RF8-O\newline

\subsubsection{UC9 - Inserimento manuale di un guasto ad una area illuminata}
\textbf{Attore:} Utente autenticato.\newline
\textbf{PRE:} è presente un impianto non funzionante che non è incluso nella lista degli impianti guasti.\newline
\textbf{POST:} l'impianto non funzionante è incluso nella lista degli impianti guasti.\newline
\textbf{Scenario principale:}
\begin{enumerate}
    \item L'utente accede al sistema;
    \item L'utente effettua il login con proprie credenziali;
    \item L'utente avvia la procedura per l’inserimento di un impianto luminoso guasto;
    \item Viene consultato l’elenco degli impianti di illuminazione attivi(include [UC10]).
    \item L'utente marca il dispositivo interessato come guasto, scatenandone l’inserimento dell’elenco degli impianti di illuminazione guasti.
\end{enumerate}
\textbf{Requisiti collegati:} RF9-O\newline

\subsubsection{UC10 - Creazione di nuova area illuminata}
\textbf{Attore:} Amministratore.\newline
\textbf{PRE:} l'area illuminata non è presente nel sistema.\newline
\textbf{POST:} l'area illuminata è presente nel sistema ed è possibile gestirla tramite applicazione.\newline
\textbf{Scenario principale:}
\begin{enumerate}
    \item L'amministratore accede al sistema;
    \item L'amministratore effettua il login con proprie credenziali;
    \item L'amministratore avvia la procedura di creazione di una nuova area illuminata;
    \item L'amministratore specifica posizione geografica e relativi dettagli;
    \item Viene ottenuta la conferma di inserimento.
\end{enumerate}
\textbf{Requisiti collegati:} RF10-O\newline

\subsubsection{UC11 - Riconfigurazione di area illuminata esistente}
\textbf{Attore:} Amministratore.\newline
\textbf{PRE:} l'area illuminata è registrata con dati arbitrari.\newline
\textbf{POST:} l'area illuminata è registrata con i dati aggiornati.\newline
\textbf{Scenario principale:}
\begin{enumerate}
    \item L'amministratore accede al sistema;
    \item L'amministratore effettua il login con proprie credenziali;
    \item L'amministratore avvia la procedura di modifica di un'area illuminata;
    \item Viene visualizzato l'elenco delle aree illuminate;
    \item L'amministratore selezione l'area illuminata che desidera modificare;
    \item Viene visualizzata la schermata di modifica dell'area illuminata selezionata in 5;
    \item L'amministratore modifica l'area illuminata con dati aggiornati;
    \item Viene ottenuta la conferma di modifica.
\end{enumerate}
\textbf{Requisiti collegati:} RF11-O\newline

\subsubsection{UC11.1 - Aggiunta di un nuovo sensore ad un'area illuminata}
\textbf{Attore:} Amministratore.\newline
\textbf{PRE:} il sensore è fisicamente presente in un'area, ma non è configurato per essere parte del sistema gestito dall'applicazione.\newline
\textbf{POST:} il sensore è inserito nel sistema ed è raggiungibile.\newline
\textbf{Scenario principale:}
\begin{enumerate}
    \item L'amministratore accede al sistema;
    \item L'amministratore effettua il login con proprie credenziali;
    \item L'amministratore avvia procedura inserimento;
    \item L'amministratore specifica tipo di interazione push/pull, dettagli, posizione geografica dispositivo, raggio d'azione;
    \item L'amministratore specifica l'area illuminata di riferimento;
    \item Viene ottenuta la conferma dell'inserimento.
\end{enumerate}
\textbf{Requisiti collegati:} RF17-O\newline

\subsubsection{UC11.2 - Rimozione di un sensore}
\textbf{Attore:} Amministratore.\newline
\textbf{PRE:} il sensore è configurato per essere parte del sistema gestito dall'applicazione.\newline
\textbf{POST:} il sensore non è più presente nel sistema.\newline
\textbf{Scenario principale:}
\begin{enumerate}
    \item L'amministratore accede al sistema;
    \item L'amministratore effettua il login con proprie credenziali;
    \item L'amministratore avvia procedura di rimozione di un sensore;
    \item Viene richiesta la conferma;
    \item L'amministratore conferma la rimozione del sensore;
    \item Viene ottenuta la conferma di rimozione.
\end{enumerate}
\textbf{Estensioni:}
\begin{itemize}
    \item [a.] L'amministratoere non conferma la rimozione alla richiesta di conferma;
    \begin{enumerate}
        \item Il sistema non subisce modifiche;
        \item L'amministratore visualizzerà l'elenco delle aree illuminate;
    \end{enumerate}
\end{itemize}
\textbf{Requisiti collegati:} RF18-O\newline

\subsubsection{UC11.3 - Associazione di un nuovo impianto di illuminazione ad un'area illuminata}
\textbf{Attore:} Amministratore.\newline
\textbf{PRE:} l'impianto è fisicamente presente ma non è registrato nel sistema.\newline
\textbf{POST:} l'impianto è stato registrato correttamente e sarà possibile gestirlo tramite applicazione.\newline
\textbf{Scenario principale:}
\begin{enumerate}
    \item L'amministratore accede al sistema;
    \item L'amministratore effettua il login con proprie credenziali;
    \item L'amministratore avvia procedura inserimento;
    \item L'amministratore specifica il sensore che gestirà l'impianto e i relativi dettagli;
    \item Viene ottenuta la conferma dell'inserimento.
\end{enumerate}
\textbf{Requisiti collegati:} RF19-O\newline

\subsubsection{UC11.4 - Rimozione di un impianto luminoso esistente}
\textbf{Attore:} Amministratore.\newline
\textbf{PRE:} l'impianto luminoso è registrato nel sistema.\newline
\textbf{POST:} l'impianto luminoso è stato cancellato dal database e non sarà possibile gestirlo dall'applicazione.\newline
\textbf{Scenario principale:}
\begin{enumerate}
    \item L'amministratore accede al sistema;
    \item L'amministratore effettua il login con proprie credenziali;
    \item L'ammiistratore avvia procedura di rimozione;
    \item Viene visualizzato l'elenco degli impianti luminosi esistenti;
    \item L'amministratore seleziona l'impianto da rimuovere dal sistema;
    \item Viene chiesta la conferma di eliminazione;
    \item L'amministratore conferma l'operazione;
    \item Viene rimosso l'impianto luminoso dal sistema;
    \item Viene ottenuta la conferma di rimozione.
\end{enumerate}
\textbf{Estensioni:}
\begin{itemize}
    \item [a.] L'amministratoere non conferma la rimozione alla richiesta di conferma;
    \begin{enumerate}
        \item Viene visualizzato l'elenco degli impianti luminosi esistenti;
        \item L'amministratore dovrà selezionare un impianto luminoso da eliminare o annullare la procedura;
    \end{enumerate}
    \item [b.] Viene annullata la procedura di rimozione;
    \begin{enumerate}
        \item La lista degli impianti luminosi non subisce modifiche;
        \item Viene visualizzata la schermata principale.
    \end{enumerate}
\end{itemize}
\textbf{Requisiti collegati:} RF20-O\newline

\subsubsection{UC12 - Rimozione di area illuminata esistente}
\textbf{Attore:} Amministratore.\newline
\textbf{PRE:} l'area illuminata è presente nel sistema e visibile tramite applicazione.\newline
\textbf{POST:} l'area illuminata non è presente nel sistema.\newline
\textbf{Scenario principale:}
\begin{enumerate}
    \item L'amministratore accede al sistema;
    \item L'amministratore effettua il login con proprie credenziali;
    \item L'amministratore avvia la procedura di rimozione di un'area illuminata;
    \item Viene visualizzato l'elenco delle aree illuminate esistenti;
    \item L'amministratore selezione l'area illuminata che desidera rimuovere;
    \item Viene visualizzata la richiesta di conferma di cancellazione;
    \item L'amministratore conferma l'operazione;
    \item Viene rimossa l'area illuminata dal sistema;
    \item Viene ottenuta la conferma di rimozione.
\end{enumerate}
\textbf{Estensioni:}
\begin{itemize}
    \item [a.] L'amministratore non conferma la rimozione alla richiesta di conferma:
    \begin{enumerate}
        \item Viene visualizzato l'elenco delle aree illuminate esistenti;
        \item L'amministratore dovrà selezionare un'area illuminata da eliminare o annullare la procedura;
    \end{enumerate}
    \item [b.] Viene annullata la procedura di rimozione:
    \begin{enumerate}
        \item La lista delle aree illuminate non subisce modifiche;
        \item Viene visualizzata la schermata principale.
    \end{enumerate}
\end{itemize}
\textbf{Requisiti collegati:} RF12-O\newline

\subsubsection{UC13 - Aumento automatico dell'intensità luminosa di un'area illuminata}
\textbf{Attori:} Sensore \newline
\textbf{PRE:} un'area luminosa configurata illumina con un'intensità iniziale arbitraria.\newline
\textbf{POST:} l'area luminosa summenzionata illumina con una precisa intensità finale.\newline
\textbf{Scenario principale:}
% DA CONTROLLARE: Non è il sensore che modifica l'intensità, ma esso fornisce il
% l'informazione al software del passaggio della persona: poi il software
% automaticamente aumenta la luminosità.
\begin{enumerate}
    \item Il sensore rileva la presenza di persone in una area illuminata precisa; [UC1]
    \item Il sistema di gestione dell'illuminazione aumenta l'intensità luminosa dell'area illuminata rilevata in 1.
\end{enumerate}
\textbf{Requisiti collegati:} RF13-O\newline

\subsubsection{UC14 - Diminuzione automatica dell'intensità luminosa di un'area illuminata}
\textbf{Attori:} Sensore \newline
\textbf{PRE:} un'area luminosa configurata illumina con un'intensità iniziale arbitraria.\newline
\textbf{POST:} l'area luminosa summenzionata illumina con una precisa intensità finale.\newline
\textbf{Scenario principale:}
% DA CONTROLLARE: Non è il sensore che modifica l'intensità, ma esso fornisce il
% l'informazione al software della mancata presenza della persona: poi il software
% automaticamente diminuisce la luminosità.
\begin{enumerate}
    \item Il sensore rileva che in un'area illuminata con intensità luminosa alta non ci sono persone presenti;
    \item il sistema di gestione dell'illuminazione diminuisce l'intensità luminosa dell'area illuminata rilevata in 1.
\end{enumerate}
\textbf{Requisiti collegati:} RF14-O\newline

\subsubsection{UC15 - Settaggio di un impianto in modalità automatica}
\textbf{Attore:} Amministratore.\newline
\textbf{PRE:} l'impianto non è stato settato con modalità automatica.\newline
\textbf{POST:} l'impianto ha la modalità automatica attiva e può gestire i dispositivi collegati ad esso.\newline
\textbf{Scenario principale:}
\begin{enumerate}
    \item L'amministratore accede al sistema;
    \item L'amministratore effettua il login con proprie credenziali;
    \item L'amministratore avvia la procedura di settaggio di un impianto in modalità automatica;
    \item L'amministratore visualizza l'elenco degli impianti esistenti;
    \item L'amministratore seleziona l'impianto che desidera impostare con modalità automatica;
    \item Viene ottenuta la conferma di attivazione della modalità automatica dell'impianto selezionato in 5.
\end{enumerate}
\textbf{Requisiti collegati:} RF15-O\newline

\subsubsection{UC16 - Rimozione di area illuminata da elenco aree illuminate con guasti}
\textbf{Attore:} Amministratore.\newline
\textbf{PRE:} l'impianto è presente nel sistema come impianto guasto.\newline
\textbf{POST:} l'impianto è presente nel sistema ma viene indicato come impianto attivo e non più come impianto guasto.\newline
\textbf{Scenario principale:}
% DA CONTROLLARE: ultimo pezzo modificato da verificare
\begin{enumerate}
    \item L'amministratore accede al sistema;
    \item L'amministratore effettua il login con proprie credenziali;
    \item L'amministratore avvia la procedura di rimozione di un impianto guasto;
    \item Viene visualizzato l'elenco degli impianti guasti esistenti;
    \item L'amministratore seleziona l'impianto che desidera rimuovere dall'elenco;
    \item Viene ottenuta la conferma di rimozione;
    \item L'impianto ritorna nella lista degli impianti attivi.
\end{enumerate}
\textbf{Requisiti collegati:} RF16-O\newline

\newpage
\section{Requisiti}
Ogni requisito è identificato da un codice la cui struttura è definita nelle \textit{Norme di Progetto}.
\subsection{Requisiti funzionali}
%R+tipologia(F/Q/V)+caso d'uso relativo.sottocaso - importanza (O/D/F) 
\setlength\tabcolsep{4pt}
\begin{longtable}{|c|p{7cm}|c|p{4cm}|}
    \hline
    \multicolumn{4}{| c |}{\textbf{Requisiti funzionali}} \\
    \hline
    \textbf{Codice} & \textbf{Descrizione} & \textbf{Rilevanza} & \textbf{Fonti} \\
    \hline
    RF1-O & Rilevamento della presenza di individui in una delle aree illuminate. & Obbligatorio & UC1 \\
    \hline
    RF2-O & Acquisizione dell'intensità luminosa e successiva determinazione precisa del livello di luminosità. & Obbligatorio & UC2 \\    
    \hline
    RF3-O & Una volta acquisito il livello di luminosità iniziale, l'operatore autenticato aumenta manualmente il livello di intesità luminosa. & Obbligatorio & UC3 \\    
    \hline
    RF4-O & Una volta acquisito il livello di luminosità iniziale, l'operatore autenticato diminuisce manualmente il livello di intensità luminosa. & Obbligatorio & UC4 \\    
    \hline
    RF5-O & Il gestore può effettuare l'accesso per gestire manualmente i sistemi di illuminazione. & Obbligatorio & UC5 \\    
    \hline
    RF6-O & Il gestore può effettuare il logout dall'interno dell'area di gestione dei sistemi. & Obbligatorio & UC6 \\
    \hline
    RF7-O & Il gestore può consultare l'intero elenco delle aree illuminate. & Obbligatorio & UC7 \\
    \hline
    RF8-O & Il gestore può consultare l'intero elenco delle aree illuminate con guasti. & Obbligatorio & UC8 \\
    \hline
    RF9-O & Il gestore può aggiungere manualmente un guasto selezionando un impianto dalla lista di quelli attivi. & Obbligatorio & UC9 \\
    \hline
    RF10-O & Il gestore può creare una nuova area illuminata, inserendone la posizione geografica e i relativi dettagli. & Obbligatorio & UC10 \\
    \hline
    RF11-O & Il gestore può modificare i dettagli di un'area illuminata già esistente. & Obbligatorio & UC11 \\
    \hline
    RF12-O & Il gestore può rimuovere un'area illuminata già esistente. & Obbligatorio & UC12 \\
    \hline
    RF13-O & Il sistema di gestione dell'illuminazione aumenta l'intensità luminosa al passaggio di una o più persone. & Obbligatorio & UC13 \\
    \hline
    RF14-O & Il sistema di gestione dell'illuminazione diminuisce l'intensità luminosa al passaggio di una o più persone. & Obbligatorio & UC14 \\
    \hline
    RF15-O & Il gestore può impostare la modalità di funzionamento automatico per l'impianto selezionato. & Obbligatorio & UC15 \\
    \hline
    RF16-O & Il gestore può rimuovere un'area illuminata dall'elenco delle aree illuminate con guasti e ritorna nelle aree illuminate attive. & Obbligatorio & UC16 \\
    \hline
    RF17-O & Inserimento di un nuovo sensore presente in un'area ma non ancora inserito per essere gestito dal sistema. & Obbligatorio & UC11.1 \\
    \hline
    RF18-O & Il gestore può rimuovere dal sistema uno dei sensori registrati. & Obbligatorio & UC11.2\\
    \hline
    RF19-O & Il gestore può inserire nel sistema l'impianto di illuminazione da attivare con i relativi dettagli. & Obbligatorio & UC11.3\\
    \hline
    RF20-O & Il gestore può rimuovere dal sistema uno degli impianti luminosi registrati. & Obbligatorio & UC11.4 \\
    \bottomrule
\end{longtable}

%\newpage
%\section{Requisiti}
%Ogni requisito è identificato da un codice la cui struttura è definita nelle \textit{Norme di Progetto}.
%\subsection{Requisiti funzionali}
%%R+tipologia(F/Q/V)+caso d'uso relativo.sottocaso - importanza (O/D/F) 
%\setlength\tabcolsep{4pt}
%\begin{longtable}{|c|p{7cm}|c|p{4cm}|}
%    \hline
%    \multicolumn{4}{| c |}{\textbf{Requisiti funzionali}} \\
%    \hline
%    \textbf{Codice} & \textbf{Descrizione} & \textbf{Rilevanza} & \textbf{Fonti} \\
%    \hline
%    RF1-O & Rilevamento della presenza di individui in una delle aree illuminate. & Obbligatorio & UC1 \\
%    \hline
%    RF2-O & Inserimento di un nuovo sensore presente in un'area ma non ancora inserito per essere gestito dal sistema. & Obbligatorio & UC2 \\
%    \hline
%    RF3-O & Acquisizione dell'intensità luminosa e successiva determinazione precisa del livello di luminosità. & Obbligatorio & UC3 \\    
%    \hline
%    RF4-O & Una volta acquisito il livello di luminosità iniziale, l'operatore autenticato aumenta manualmente il livello di intesità luminosa. & Obbligatorio & UC4 \\    
%    \hline
%    RF5-O & Una volta acquisito il livello di luminosità iniziale, l'operatore autenticato diminuisce manualmente il livello di intensità luminosa. & Obbligatorio & UC5 \\    
%    \hline
%    RF6-O & Il gestore può effettuare l'accesso per gestire manualmente i sistemi di illuminazione. & Obbligatorio & UC6 \\    
%    \hline
%    RF7-O & Il gestore può effettuare il logout dall'interno dell'area di gestione dei sistemi. & Obbligatorio & UC7 \\
%    \hline
%    RF8-O & Il gestore può inserire nel sistema l'impianto da attivare con i
%    relativi dettagli. & Obbligatorio & UC8\\
%    \hline
%    RF9-O & Il gestore può rimuovere dal sistema uno degli impianti luminosi
%    registrati. & Obbligatorio & UC9 \\
%    \hline
%    RF10-O & Il gestore può consultare l'intero elenco degli impianti attivi. & Obbligatorio & UC10 \\
%    \hline
%    RF11-O & Il gestore può consultare l'intero elenco degli impianti guasti. & Obbligatorio & UC11 \\
%    \hline
%    RF12-O & Il gestore può aggiungere manualmente un guasto selezionando un
%    impianto dalla lista di quelli attivi. & Obbligatorio & UC12 \\
%    \hline
%    RF13-O & Il gestore può consultare l'elenco delle aree illuminate. & Obbligatorio & UC13 \\
%    \hline
%    RF14-O & Il gestore può creare una nuova area illuminata, inserendone la
%    posizione geografica e i relativi dettagli. & Obbligatorio & UC14 \\
%    \hline
%    RF15-O & Il gestore può modificare i dettagli di un'area illuminata già esistente. & Obbligatorio & UC15 \\
%    \hline
%    RF16-O & Il gestore può rimuovere un'area illuminata già esistente. &
%    Obbligatorio & UC16 \\
%    \hline
%    RF17-O & Il sistema di gestione dell'illuminazione aumenta l'intensità
%    luminosa al passaggio di una o più persone. & Obbligatorio & UC17 \\
%    \hline
%    RF18-O & Il sistema di gestione dell'illuminazione diminuisce l'intensità
%    luminosa al passaggio di una o più persone. & Obbligatorio & UC18 \\
%    \hline
%    RF19-O & Il gestore può impostare la modalità di funzionamento automatico
%    per l'impianto selezionato. & Obbligatorio & UC19 \\
%    \hline
%    RF20-O & Il gestore può rimuovere un impianto dall'elenco degli impianti
%    guasti e ritorna negli impianti attivi. & Obbligatorio & UC20 \\
%    \bottomrule
%\end{longtable}
\end{document}
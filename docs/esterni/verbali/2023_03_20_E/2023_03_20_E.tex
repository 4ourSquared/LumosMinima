\documentclass[a4paper, 12pt]{article}

\newcommand{\templates} {../../../template}
\usepackage[a4paper, margin=2.5cm]{geometry}
\usepackage{fancyvrb}
\usepackage{stackengine}
\usepackage{lastpage}

\usepackage{enumitem}
\setlist[itemize]{noitemsep}
\setlist[enumerate]{noitemsep}

\let\oldpar\paragraph
\renewcommand{\paragraph}[1]{\oldpar{#1\\}\noindent}

% Avoid dots in the table of contents, it mess with the gulpease calculation
\makeatletter
\renewcommand{\@dotsep}{10000} 
\makeatother

\newcommand{\makeindexdetails}{
	\pagestyle{fancy}
	\lhead{4ourSquared} \chead{\includegraphics[width=1cm]{../../template/4ourSquared_logo} } \rhead{Versione e Indice}
	\pagenumbering{Roman}
}

\newcommand{\makecontentsdetails}[1]{
	\clearpage
    \renewcommand{\footrulewidth}{0.4pt}
	\pagestyle{fancy}
	\lhead{4ourSquared} \chead{\includegraphics[width=1cm]{../../template/4ourSquared_logo}} \rhead{\nouppercase{\leftmark}}
	\pagenumbering{arabic}
    \lfoot{#1} \cfoot{} \rfoot{\thepage / \pageref{LastPage}}
}
\usepackage{graphicx}
\usepackage{hyperref}
\usepackage{makecell}

\newcommand{\settitolo}[1]{\newcommand{\titolo}{#1\\}}
\newcommand{\setprogetto}[1]{\newcommand{\progetto}{#1\\}}
\newcommand{\setcommittenti}[1]{\newcommand{\committenti}{#1\\}}
\newcommand{\setredattori}[1]{\newcommand{\redattori}{#1\\}}
\newcommand{\setrevisori}[1]{\newcommand{\revisori}{#1\\}}
\newcommand{\setresponsabili}[1]{\newcommand{\responsabili}{#1\\}}
\newcommand{\setversione}[1]{
	\ifdefined\versione\renewcommand{\versione}{#1\\}
	\else\newcommand{\versione}{#1\\}\fi
}
\newcommand{\setdestuso}[1]{\newcommand{\uso}{#1\\}}
\newcommand{\setdescrizione}[1]{\newcommand{\descrizione}{#1\\}}

\newcommand{\makefrontpage}{
	\begin{titlepage}
		\begin{center}

		\includegraphics[width=0.4\textwidth]{\templates/4ourSquared_logo}\\

		{\Large 4OURSQUARED}\\[6pt]
		\href{mailto://4oursquared.unipd@gmail.com}{4oursquared.unipd@gmail.com}\\
		
		\ifdefined\progetto
		\vspace{1cm}
		{\Large\progetto}
		{\large\committenti}
		\else\fi
		
		\vspace{1.5cm}
		{\LARGE\titolo}
		
		\vfill
		
		\begin{tabular}{r | l}
		\multicolumn{2}{c}{\textit{Informazioni}}\\
		\hline
		
		\ifdefined\redattori
			\textit{Redattori} &
			\makecell[l]{\redattori}\\
		\else\fi
		\ifdefined\revisori
			\textit{Revisori} &
			\makecell[l]{\revisori}\\
		\else\fi
		\ifdefined\responsabili
			\textit{Responsabili} &
			\makecell[l]{\responsabili}\\
		\else\fi
		
		\ifdefined\versione
			\textit{Versione} & \versione
		\else\fi
		
		\textit{Uso} & \uso
		
		\end{tabular}
		
		\vspace{2cm}
		
		\ifdefined\descrizione
		Descrizione
		\vspace{6pt}
		\hrule
		\descrizione
		\else\fi
		\end{center}
	\end{titlepage}
}
\usepackage{hyperref}
\usepackage{array}
\usepackage{tabularx}
\usepackage{adjustbox}

\newcounter{verscount}
\setcounter{verscount}{0}
\newcommand{\addversione}[5]{
	\ifdefined\setversione
		\setversione{#1}
	\else\fi
	\stepcounter{verscount}
	\expandafter\newcommand%
		\csname ver\theverscount \endcsname{#1&#2&#3&#4&#5}
}

\newcommand{\listversioni}{
	\ifnum\value{verscount}>1
		\csname ver\theverscount \endcsname
		\addtocounter{verscount}{-1}
		\\\hline
		\listversioni
	\else
		\csname ver\theverscount \endcsname\\\hline
	\fi
}

\newcommand{\makeversioni}{
	\begin{center}
		\begin{tabularx}{\textwidth}{|c|c|c|c|X|}
		\hline
		\textbf{Versione} & \textbf{Data} & \textbf{Redattore} & \textbf{Verificatore} & \textbf{Descrizione} \\
		\hline
		\listversioni
		\end{tabularx}
	\end{center}
	\clearpage
}

\settitolo{Verbale del 20/03/2023 con SyncLab}
\setredattori{Brotto Romina \\ Soldà Matteo}
\setrevisori{Alberti Nicolas \\ Ceccato Francesco}
\setdestuso{esterno}
\setdescrizione{
    Verbale dell'incontro del 20/03/2023 con SyncLab
}

\begin{document}
\makefrontpage
\section*{Informazioni Generali}
\begin{itemize}
    \item Data e Ora: 20/03/2022 15:00
    \item Durata: 0.45h
    \item Piattaforma: Google Meet
    \item Partecipanti
    \begin{itemize}
        \item Alberti Nicolas
        \item Brotto Romina
        \item Cavaliere Erica
        \item Ceccato Francesco
        \item Salami Lorenzo
        \item Soldà Matteo
        \item Galvagni Matteo (SyncLab)
        \item Pallaro Fabio (SyncLab)
    \end{itemize}
\end{itemize}
\section*{O.D.G}
\begin{itemize}
    \item Presentazioni ed Esposizione del Progetto
    \item Q\&A
    \item Varie ed Eventuali
\end{itemize}
\section*{Conclusioni}
\begin{itemize}
    \item Dalla riunione, come già comunque anticipato nella presentazione web del capitolato, esso si presenta come una piattaforma \textit{blockchain-based} che si pone come obiettivo quello di evitare il fenomeno del \textit{review bombing} e della censura e/o corruzione delle recensioni.
    \item Al seguito della prima presentazione, sono state poste le seguenti domande (con annessa risposta):
    \begin{enumerate}
        \item \textbf{Collegamento tra una transazione e la rispettiva recensione} \\ Il venditore del servizio/prodotto non necessita di un modulo da inserire sul suo sito, in quanto il tutto si basa su di un contratto digitale. La piattaforma risulta quindi indipendente rispetto agli attori stessi.
        \item \textbf{Sviluppo dei pagamenti} \\ Il pagamento, nel contesto del progetto, dovrà essere generato, ma la sua importanza è marginale. Essenzialmente, dopo aver effettuato il pagamento, ci sarà un \textit{redirect} sulla piattaforma che in base al tipo di contratto permetterà di scrivere la recensione al momento adeguato.
        \item \textbf{Tempistica per la scrittura della recensione} \\ Dipende, in base al tipo di contratto, la recensione potrà essere scritta subito, entro un cento limite di tempo oppure dopo un determinato periodo di tempo.
        \item \textbf{Fondamenti di esistenza del progetto} \\ L'utilizzo di determinate tecnologie quali \textit{Solidity} e \textit{Metamask} è dovuta al fatto che entrambi funzionano sulle blockchain \textit{ethereum-compatibili}, considerando inoltre che esse rappresentano gli attuali standard \textit{de facto}.
        \item \textbf{Supporto dell'azienda} \\ L'azienda riferisce di poter inserire tutti i membri del gruppo in un server \textit{Discord} nel quale sarà possibile richiedere supporto in caso di reale necessità. Per quanto riguarda invece problemi di entità maggiore, si posso impegnare per delle call bi-settimanali. L'azienda ci tiene a sottolineare che, essendo la proponente del progetto, essi non daranno troppi aiuti sullo sviluppo in quanto il progetto sarebbe da noi svolto \textbf{per} loro e non \textbf{con} loro.
        \item \textbf{Licenza del codice sorgente} \\ L'azienda afferma che qualora ci impegnassimo con il loro progetto, il codice sorgente risulterebbe di nostra proprietà e quindi la scelta della licenza sarà di nostra discrezione senza vincoli imposti.
    \end{enumerate}
    \item Nel complesso, il gruppo di lavoro si considera soddisfatto del colloquio, esprimendo interesse per il capitolato e per l'azienda che si è dimostrata competente, cortese e disponibile.
\end{itemize}
\end{document}
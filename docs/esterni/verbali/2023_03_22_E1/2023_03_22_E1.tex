\documentclass[a4paper, 12pt]{article}

\newcommand{\templates} {../../../template}
\usepackage[a4paper, margin=2.5cm]{geometry}
\usepackage{fancyvrb}
\usepackage{stackengine}
\usepackage{lastpage}

\usepackage{enumitem}
\setlist[itemize]{noitemsep}
\setlist[enumerate]{noitemsep}

\let\oldpar\paragraph
\renewcommand{\paragraph}[1]{\oldpar{#1\\}\noindent}

% Avoid dots in the table of contents, it mess with the gulpease calculation
\makeatletter
\renewcommand{\@dotsep}{10000} 
\makeatother

\newcommand{\makeindexdetails}{
	\pagestyle{fancy}
	\lhead{4ourSquared} \chead{\includegraphics[width=1cm]{../../template/4ourSquared_logo} } \rhead{Versione e Indice}
	\pagenumbering{Roman}
}

\newcommand{\makecontentsdetails}[1]{
	\clearpage
    \renewcommand{\footrulewidth}{0.4pt}
	\pagestyle{fancy}
	\lhead{4ourSquared} \chead{\includegraphics[width=1cm]{../../template/4ourSquared_logo}} \rhead{\nouppercase{\leftmark}}
	\pagenumbering{arabic}
    \lfoot{#1} \cfoot{} \rfoot{\thepage / \pageref{LastPage}}
}
\usepackage{graphicx}
\usepackage{hyperref}
\usepackage{makecell}

\newcommand{\settitolo}[1]{\newcommand{\titolo}{#1\\}}
\newcommand{\setprogetto}[1]{\newcommand{\progetto}{#1\\}}
\newcommand{\setcommittenti}[1]{\newcommand{\committenti}{#1\\}}
\newcommand{\setredattori}[1]{\newcommand{\redattori}{#1\\}}
\newcommand{\setrevisori}[1]{\newcommand{\revisori}{#1\\}}
\newcommand{\setresponsabili}[1]{\newcommand{\responsabili}{#1\\}}
\newcommand{\setversione}[1]{
	\ifdefined\versione\renewcommand{\versione}{#1\\}
	\else\newcommand{\versione}{#1\\}\fi
}
\newcommand{\setdestuso}[1]{\newcommand{\uso}{#1\\}}
\newcommand{\setdescrizione}[1]{\newcommand{\descrizione}{#1\\}}

\newcommand{\makefrontpage}{
	\begin{titlepage}
		\begin{center}

		\includegraphics[width=0.4\textwidth]{\templates/4ourSquared_logo}\\

		{\Large 4OURSQUARED}\\[6pt]
		\href{mailto://4oursquared.unipd@gmail.com}{4oursquared.unipd@gmail.com}\\
		
		\ifdefined\progetto
		\vspace{1cm}
		{\Large\progetto}
		{\large\committenti}
		\else\fi
		
		\vspace{1.5cm}
		{\LARGE\titolo}
		
		\vfill
		
		\begin{tabular}{r | l}
		\multicolumn{2}{c}{\textit{Informazioni}}\\
		\hline
		
		\ifdefined\redattori
			\textit{Redattori} &
			\makecell[l]{\redattori}\\
		\else\fi
		\ifdefined\revisori
			\textit{Revisori} &
			\makecell[l]{\revisori}\\
		\else\fi
		\ifdefined\responsabili
			\textit{Responsabili} &
			\makecell[l]{\responsabili}\\
		\else\fi
		
		\ifdefined\versione
			\textit{Versione} & \versione
		\else\fi
		
		\textit{Uso} & \uso
		
		\end{tabular}
		
		\vspace{2cm}
		
		\ifdefined\descrizione
		Descrizione
		\vspace{6pt}
		\hrule
		\descrizione
		\else\fi
		\end{center}
	\end{titlepage}
}
\usepackage{hyperref}
\usepackage{array}
\usepackage{tabularx}
\usepackage{adjustbox}

\newcounter{verscount}
\setcounter{verscount}{0}
\newcommand{\addversione}[5]{
	\ifdefined\setversione
		\setversione{#1}
	\else\fi
	\stepcounter{verscount}
	\expandafter\newcommand%
		\csname ver\theverscount \endcsname{#1&#2&#3&#4&#5}
}

\newcommand{\listversioni}{
	\ifnum\value{verscount}>1
		\csname ver\theverscount \endcsname
		\addtocounter{verscount}{-1}
		\\\hline
		\listversioni
	\else
		\csname ver\theverscount \endcsname\\\hline
	\fi
}

\newcommand{\makeversioni}{
	\begin{center}
		\begin{tabularx}{\textwidth}{|c|c|c|c|X|}
		\hline
		\textbf{Versione} & \textbf{Data} & \textbf{Redattore} & \textbf{Verificatore} & \textbf{Descrizione} \\
		\hline
		\listversioni
		\end{tabularx}
	\end{center}
	\clearpage
}

\settitolo{Verbale del 22/03/2023 con Imola Informatica}
\setredattori{Alberti Nicolas \\ Brotto Romina \\ Soldà Matteo}
\setrevisori{Cavaliere Erica \\ Ceccato Francesco}
\setdestuso{esterno}
\setdescrizione{
    Verbale dell'incontro del 22/03/2023 con Imola Informatica
}

\begin{document}
\makefrontpage
\section*{Informazioni Generali}
\begin{itemize}
    \item Data e Ora: 22/03/2022 15:00
    \item Durata: 0.25h
    \item Piattaforma: Google Meet
    \item Partecipanti
    \begin{itemize}
        \item Alberti Nicolas
        \item Brotto Romina
        \item Cavaliere Erica
        \item Ceccato Francesco
        \item Salami Lorenzo
        \item Soldà Matteo
        \item Patera Lorenzo (Imola Informatica)
    \end{itemize}
\end{itemize}
\section*{O.D.G}
\begin{itemize}
    \item Q\&A
    \item Varie ed Eventuali
\end{itemize}
\section*{Conclusioni}
Dal presente colloquio sono emerse le seguenti domande poste dal gruppo di lavoro, con annessa risposta del Sig. Patera:
\begin{itemize}
    \item \textbf{Copertura dei test} \\ Andranno utilizzati dei tool per verificare la copertura del codice (\textit{es: SonarQube}). Il risultato di questi test deve essere $\geq 80\%$.
    \item \textbf{Invertibilità UC3} \\ L'invertibilità del caso d'uso citato è facoltativo in quanto non strettamente necessario.
    \item \textbf{Issues tracking automatico} \\ Generalmente, qualora risultasse che uno dei dispositivi non rispondesse ad un \textit{ping} o, per esempio, una fotocellula non rilevasse l'accensione di una/più luce/i, allora basterà aprire un \textit{ticket} tramite un sistema di \textit{ticketing} quale a titolo esemplificativo \textit{Jira}. La risoluzione e la chiusura non rappresenterebbero una nostra competenza.
    \item \textbf{Licenza} \\ L'azienda consiglia di utilizzare una licenza \textit{open-source non commerciale} quale \textit{GNU GPL3}.
    \item \textbf{Supporto} \\ L'azienda si può impegnare ad aggiungere tutti i membri del gruppo in un server \textit{Discord} e a fornire una chat \textit{Telegram} dove poter rispondere a quesiti veloci o di facile risoluzione. Qualora il problema risultasse di maggiore entità, si può richiedere tramite \textit{Discord} o \textit{Telegram} una \textit{call} che sarà poi richiesta in maniera ufficiale tramite comunicazione mail.
    \item \textbf{Utilizzo di telecamere e Privacy} \\ Le fotocellule fornite non dispongono di fotocamere o videocamere, per questo motivo non sorgono problemi di privacy.
\end{itemize}
Il Sig. Patera ha inoltre aggiunto i seguenti punti a quanto sopra riportato:
\begin{itemize}
    \item A causa di una mancanza di chip, le luci e i sensori non sono fisicamente disponibili, per questo si svolgerà il progetto in un ambiente virtuale fornito dalla azienda e che sarà disponibile $24/7$.
    \item Qualora si rendesse necessario, l'azienda si occuperà di organizzare dei piccoli seminari per entrare in confidenza con le tecnologie in loro possesso e che risulterebbero di interesse per lo sviluppo del progetto (\textit{es: seminario sulle API REST del server su cui avverrà il deploy del progetto}).
    \item Le tecnologie e i linguaggi da utilizzare sono a libera scelta. Su richiesta, l'azienda può consigliare quali utilizzare, così da poter fornire un maggior supporto.
\end{itemize}
Nel complesso, il gruppo di lavoro si dimostra molto interessato al progetto, in quanto presenta una buona libertà di scelta per quanto riguarda le tecnologie da utilizzare e la possibilità di interfacciarsi con componenti tangibili. L'azienda, nel complesso, si è dimostrata disponibile, cortese e competente. Riteniamo inoltre che sia di grande valore la possibilità di seguire dei seminari tenuti da esperti del settore.
\end{document}
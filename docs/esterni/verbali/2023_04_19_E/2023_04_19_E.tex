\documentclass[a4paper, 12pt]{article}

\newcommand{\templates} {../../../template}
\usepackage[a4paper, margin=2.5cm]{geometry}
\usepackage{fancyvrb}
\usepackage{stackengine}
\usepackage{lastpage}

\usepackage{enumitem}
\setlist[itemize]{noitemsep}
\setlist[enumerate]{noitemsep}

\let\oldpar\paragraph
\renewcommand{\paragraph}[1]{\oldpar{#1\\}\noindent}

% Avoid dots in the table of contents, it mess with the gulpease calculation
\makeatletter
\renewcommand{\@dotsep}{10000} 
\makeatother

\newcommand{\makeindexdetails}{
	\pagestyle{fancy}
	\lhead{4ourSquared} \chead{\includegraphics[width=1cm]{../../template/4ourSquared_logo} } \rhead{Versione e Indice}
	\pagenumbering{Roman}
}

\newcommand{\makecontentsdetails}[1]{
	\clearpage
    \renewcommand{\footrulewidth}{0.4pt}
	\pagestyle{fancy}
	\lhead{4ourSquared} \chead{\includegraphics[width=1cm]{../../template/4ourSquared_logo}} \rhead{\nouppercase{\leftmark}}
	\pagenumbering{arabic}
    \lfoot{#1} \cfoot{} \rfoot{\thepage / \pageref{LastPage}}
}
\usepackage{graphicx}
\usepackage{hyperref}
\usepackage{makecell}

\newcommand{\settitolo}[1]{\newcommand{\titolo}{#1\\}}
\newcommand{\setprogetto}[1]{\newcommand{\progetto}{#1\\}}
\newcommand{\setcommittenti}[1]{\newcommand{\committenti}{#1\\}}
\newcommand{\setredattori}[1]{\newcommand{\redattori}{#1\\}}
\newcommand{\setrevisori}[1]{\newcommand{\revisori}{#1\\}}
\newcommand{\setresponsabili}[1]{\newcommand{\responsabili}{#1\\}}
\newcommand{\setversione}[1]{
	\ifdefined\versione\renewcommand{\versione}{#1\\}
	\else\newcommand{\versione}{#1\\}\fi
}
\newcommand{\setdestuso}[1]{\newcommand{\uso}{#1\\}}
\newcommand{\setdescrizione}[1]{\newcommand{\descrizione}{#1\\}}

\newcommand{\makefrontpage}{
	\begin{titlepage}
		\begin{center}

		\includegraphics[width=0.4\textwidth]{\templates/4ourSquared_logo}\\

		{\Large 4OURSQUARED}\\[6pt]
		\href{mailto://4oursquared.unipd@gmail.com}{4oursquared.unipd@gmail.com}\\
		
		\ifdefined\progetto
		\vspace{1cm}
		{\Large\progetto}
		{\large\committenti}
		\else\fi
		
		\vspace{1.5cm}
		{\LARGE\titolo}
		
		\vfill
		
		\begin{tabular}{r | l}
		\multicolumn{2}{c}{\textit{Informazioni}}\\
		\hline
		
		\ifdefined\redattori
			\textit{Redattori} &
			\makecell[l]{\redattori}\\
		\else\fi
		\ifdefined\revisori
			\textit{Revisori} &
			\makecell[l]{\revisori}\\
		\else\fi
		\ifdefined\responsabili
			\textit{Responsabili} &
			\makecell[l]{\responsabili}\\
		\else\fi
		
		\ifdefined\versione
			\textit{Versione} & \versione
		\else\fi
		
		\textit{Uso} & \uso
		
		\end{tabular}
		
		\vspace{2cm}
		
		\ifdefined\descrizione
		Descrizione
		\vspace{6pt}
		\hrule
		\descrizione
		\else\fi
		\end{center}
	\end{titlepage}
}
\usepackage{hyperref}
\usepackage{array}
\usepackage{tabularx}
\usepackage{adjustbox}

\newcounter{verscount}
\setcounter{verscount}{0}
\newcommand{\addversione}[5]{
	\ifdefined\setversione
		\setversione{#1}
	\else\fi
	\stepcounter{verscount}
	\expandafter\newcommand%
		\csname ver\theverscount \endcsname{#1&#2&#3&#4&#5}
}

\newcommand{\listversioni}{
	\ifnum\value{verscount}>1
		\csname ver\theverscount \endcsname
		\addtocounter{verscount}{-1}
		\\\hline
		\listversioni
	\else
		\csname ver\theverscount \endcsname\\\hline
	\fi
}

\newcommand{\makeversioni}{
	\begin{center}
		\begin{tabularx}{\textwidth}{|c|c|c|c|X|}
		\hline
		\textbf{Versione} & \textbf{Data} & \textbf{Redattore} & \textbf{Verificatore} & \textbf{Descrizione} \\
		\hline
		\listversioni
		\end{tabularx}
	\end{center}
	\clearpage
}

\settitolo{Verbale del 19/04/2023 con Imola Informatica}
\setredattori{Soldà Matteo}
\setrevisori{Brotto Romina \\ Cavaliere Erica}
\setdestuso{esterno}
\setdescrizione{
    Verbale dell'incontro del 19/04/2023 con Imola Informatica
}

\begin{document}
\makefrontpage
\section*{Informazioni Generali}
\begin{itemize}
    \item Data e Ora: 19/04/2023 15:00
    \item Durata: 0.35h
    \item Piattaforma: Google Meet
    \item Partecipanti
    \begin{itemize}
        \item Alberti Nicolas
        \item Brotto Romina
        \item Cavaliere Erica
        \item Ceccato Francesco
        \item Salami Lorenzo
        \item Soldà Matteo
        \item Patera Lorenzo (Imola Informatica)
    \end{itemize}
\end{itemize}
\section*{O.D.G}
\begin{itemize}
    \item Specifiche Tecniche Generali
    \item Server di Test
    \item Varie ed Eventuali
\end{itemize}
\section*{Conclusioni}
A seguito della riunione effettuata, è emerso quanto di seguito:
\begin{itemize}
    \item Per quanto riguarda il \textbf{frontend}, ci è stato consigliato di usare \textit{ReactJS}
    \item Per quanto riguarda il \textbf{backend}, ci è stato consigliato di usare \textit{NodeJS}
    \item Per quanto riguarda la \textbf{comunicazione} tra frontend e backend ci è stato consigliato l'utilizzo di \href{https://devacademy.it/rest-api-cosa-sono-come-funzionano-e-come-progettarle/}{\textit{API REST}} o di \textit{MQTT}
    \item Per quanto riguarda i \textbf{test}, ci è stato fornito uno script Python di base per la generazione di luci virtuali sulle quali testare le comunicazioni
    \item Per quanto riguarda l'\textbf{accessibilità}, non è richiesto un livello minimo
    \item Per quanto riguarda il \textbf{software di ticketing}, ci è stato consigliato \href{redmine.org}{\textit{redmine}}
\end{itemize}
Ci è stato inoltre ricordato che, affinché il progetto sia considerato di buon livello, esso deve essere quanto più modulare possibile. Questa caratteristica può essere verificata tramite \href{https://docs.sonarqube.org/latest/}{programmi appositi} che verificano il \textit{code coverage}. Inoltre, qualora utilizzassimo le \textit{API REST}, ci è stato consigliato di utilizzare il tool \href{https://chrome.google.com/webstore/detail/talend-api-tester-free-ed/aejoelaoggembcahagimdiliamlcdmfm?authuser=2}{\textit{Talend Api Tester}}.\newline
Infine, ci è stato riferito che quando il progetto uscirà dallo stato embrionale, sarà possibile testarlo su di un server privato in possesso dell'azienda, accessibile tramite una VPN specifica.

\section*{Impegni Assunti}
I membri del gruppo, a seguito della riunione, aggiorneranno i documenti necessari e richiesti per la creazione di un \textit{Way of Working} solido con i nuovi dati, rimandandosi a lunedì 24/04 per l'eventuale correzione e successiva validazione e approvazione degli stessi.

\end{document}
\documentclass[a4paper, 12pt]{article}
\usepackage{eurosym}
\usepackage{pdflscape}
\usepackage{pgfgantt}
\usepackage{pgfplots}


\newcommand{\templates}{../../template}
\usepackage[a4paper, margin=2.5cm]{geometry}
\usepackage{fancyvrb}
\usepackage{stackengine}
\usepackage{lastpage}

\usepackage{enumitem}
\setlist[itemize]{noitemsep}
\setlist[enumerate]{noitemsep}

\let\oldpar\paragraph
\renewcommand{\paragraph}[1]{\oldpar{#1\\}\noindent}

% Avoid dots in the table of contents, it mess with the gulpease calculation
\makeatletter
\renewcommand{\@dotsep}{10000} 
\makeatother

\newcommand{\makeindexdetails}{
	\pagestyle{fancy}
	\lhead{4ourSquared} \chead{\includegraphics[width=1cm]{../../template/4ourSquared_logo} } \rhead{Versione e Indice}
	\pagenumbering{Roman}
}

\newcommand{\makecontentsdetails}[1]{
	\clearpage
    \renewcommand{\footrulewidth}{0.4pt}
	\pagestyle{fancy}
	\lhead{4ourSquared} \chead{\includegraphics[width=1cm]{../../template/4ourSquared_logo}} \rhead{\nouppercase{\leftmark}}
	\pagenumbering{arabic}
    \lfoot{#1} \cfoot{} \rfoot{\thepage / \pageref{LastPage}}
}
\usepackage{graphicx}
\usepackage{hyperref}
\usepackage{makecell}

\newcommand{\settitolo}[1]{\newcommand{\titolo}{#1\\}}
\newcommand{\setprogetto}[1]{\newcommand{\progetto}{#1\\}}
\newcommand{\setcommittenti}[1]{\newcommand{\committenti}{#1\\}}
\newcommand{\setredattori}[1]{\newcommand{\redattori}{#1\\}}
\newcommand{\setrevisori}[1]{\newcommand{\revisori}{#1\\}}
\newcommand{\setresponsabili}[1]{\newcommand{\responsabili}{#1\\}}
\newcommand{\setversione}[1]{
	\ifdefined\versione\renewcommand{\versione}{#1\\}
	\else\newcommand{\versione}{#1\\}\fi
}
\newcommand{\setdestuso}[1]{\newcommand{\uso}{#1\\}}
\newcommand{\setdescrizione}[1]{\newcommand{\descrizione}{#1\\}}

\newcommand{\makefrontpage}{
	\begin{titlepage}
		\begin{center}

		\includegraphics[width=0.4\textwidth]{\templates/4ourSquared_logo}\\

		{\Large 4OURSQUARED}\\[6pt]
		\href{mailto://4oursquared.unipd@gmail.com}{4oursquared.unipd@gmail.com}\\
		
		\ifdefined\progetto
		\vspace{1cm}
		{\Large\progetto}
		{\large\committenti}
		\else\fi
		
		\vspace{1.5cm}
		{\LARGE\titolo}
		
		\vfill
		
		\begin{tabular}{r | l}
		\multicolumn{2}{c}{\textit{Informazioni}}\\
		\hline
		
		\ifdefined\redattori
			\textit{Redattori} &
			\makecell[l]{\redattori}\\
		\else\fi
		\ifdefined\revisori
			\textit{Revisori} &
			\makecell[l]{\revisori}\\
		\else\fi
		\ifdefined\responsabili
			\textit{Responsabili} &
			\makecell[l]{\responsabili}\\
		\else\fi
		
		\ifdefined\versione
			\textit{Versione} & \versione
		\else\fi
		
		\textit{Uso} & \uso
		
		\end{tabular}
		
		\vspace{2cm}
		
		\ifdefined\descrizione
		Descrizione
		\vspace{6pt}
		\hrule
		\descrizione
		\else\fi
		\end{center}
	\end{titlepage}
}
\usepackage{hyperref}
\usepackage{array}
\usepackage{tabularx}
\usepackage{adjustbox}

\newcounter{verscount}
\setcounter{verscount}{0}
\newcommand{\addversione}[5]{
	\ifdefined\setversione
		\setversione{#1}
	\else\fi
	\stepcounter{verscount}
	\expandafter\newcommand%
		\csname ver\theverscount \endcsname{#1&#2&#3&#4&#5}
}

\newcommand{\listversioni}{
	\ifnum\value{verscount}>1
		\csname ver\theverscount \endcsname
		\addtocounter{verscount}{-1}
		\\\hline
		\listversioni
	\else
		\csname ver\theverscount \endcsname\\\hline
	\fi
}

\newcommand{\makeversioni}{
	\begin{center}
		\begin{tabularx}{\textwidth}{|c|c|c|c|X|}
		\hline
		\textbf{Versione} & \textbf{Data} & \textbf{Redattore} & \textbf{Verificatore} & \textbf{Descrizione} \\
		\hline
		\listversioni
		\end{tabularx}
	\end{center}
	\clearpage
}

\settitolo{Piano di Progetto}
\setredattori{Erica Cavaliere}
\setdestuso{esterno}
\setdescrizione{
Questo documento serve a tracciare l'efficienza del progetto. Tiene traccia dei costi sostenuti fino ad oggi e li confronta con i costi preventivati, in relazione agli obiettivi fissati.
}

\addversione{0.0.0}{22/04/2023}{Erica Cavaliere}{Redazione}{Stesura iniziale}

\def\pgfcalendarmonthitname#1{%
\ifcase#1 \or Gennaio\or Febbraio\or Marzo\or Aprile\or Maggio\or Giugno\or Luglio\or Agosto\or Settembre\or Ottobre\or Novembre\or Dicembre\fi%
}
\usepgfplotslibrary{dateplot}

\begin{document}
\makeindexdetails
\makefrontpage \makeversioni
\tableofcontents
\newpage
\clearpage
\makecontentsdetails{Piano di Progetto} 

\section{Analisi dei rischi}

\subsection{Rischi tecnologici}
Le tecnologie consigliate per lo svolgimento del progetto da parte del richiedente possono risultare nuove per alcuni componenti del gruppo, questo può portare difficoltà nell'utilizzo delle tecnologie, portando di conseguenza errori nel codice e ritardi di consegna. \newline
Per prefissare le scadenze bisognerà tenere conto del tempo impiegato per imparare le nuove tecnologie, oltre al tempo di sviluppo del progetto stesso.

\subsection{Rischi organizzativi}
Durante lo svolgimento del progetto, è possibile andare incontro a dei problemi di organizzazione degli orari e quini di non riuscire a rispettare la stima delle ore richiesta e, di conseguenza, sforare i costi indicati dal preventivo iniziale.

\section{Pianificazione del lavoro}


\paragraph{Grafico di Gantt}

\section{Preventivo e consuntivo delle ore e dei costi}

\subsection{Suddivisione delle ore previste}

\begin{tabular}{|c|c|c|c|c|c|c|c|}
    \hline
    \textbf{} & \textbf{Respon.} & \textbf{Amministr.} & \textbf{Analista} & \textbf{Proget.} & \textbf{Program.} & \textbf{Verif.} & \textbf{Totale}\\
    \hline
    Erica & 11 & 9 & 17 & 21 & 22 & 15 & 95\\
    \hline
    Francesco & 12 & 8 & 16 & 22 & 22 & 15 & 95\\
    \hline
    Lorenzo & 12 & 8 & 16 & 22 & 22 & 15 & 95\\
    \hline
    Matteo & 12 & 8 & 17 & 22 & 21 & 15 & 95\\
    \hline
    Nicolas & 11 & 9 & 17 & 22 & 21 & 15 & 95\\
    \hline
    Romina & 12 & 8 & 17 & 21 & 22 & 15 & 95\\
    \hline
    Totale & 70 & 50 & 100 & 130 & 130 & 90 & 570\\
    \hline
\end{tabular}\\[8pt]

\subsection{Costi attesi}

\begin{tabular}{|c|c|c|c|}
    \hline
    \textbf{Ruolo} & \textbf{Costo orario (\euro)} & \textbf{Ore} & \textbf{Prezzo (\euro)}\\
    \hline
    Responsabile & 30 & 70 & 2100\\
    \hline
    Amministratore & 20 & 50 & 1000\\
    \hline
    Analista & 25 & 100 & 2500\\
    \hline
    Progettista & 25 & 130 & 3250\\
    \hline
    Programmatore & 15 & 130 & 1950\\
    \hline
    Verificatore & 15 & 90 & 1350\\
    \hline\hline
    \multicolumn{2}{|c|}{\textbf{Scadenza consegna}} & \textbf{Ore tot.} & \textbf{Preventivo finale (\euro)}\\
    \hline
    \multicolumn{2}{|c|}{15/09/2023} & 570 & 12150\\
    \hline
\end{tabular}\\[8pt]

\paragraph{Grafico dei costi attesi}
\graphicspath{ {./immagini/} }
\includegraphics{grafico_costi.png}

\end{document}
\documentclass[a4paper, 12pt]{article}

\newcommand{\templates}{../../template}
\usepackage[a4paper, margin=2.5cm]{geometry}
\usepackage{fancyvrb}
\usepackage{stackengine}
\usepackage{lastpage}

\usepackage{enumitem}
\setlist[itemize]{noitemsep}
\setlist[enumerate]{noitemsep}

\let\oldpar\paragraph
\renewcommand{\paragraph}[1]{\oldpar{#1\\}\noindent}

% Avoid dots in the table of contents, it mess with the gulpease calculation
\makeatletter
\renewcommand{\@dotsep}{10000} 
\makeatother

\newcommand{\makeindexdetails}{
	\pagestyle{fancy}
	\lhead{4ourSquared} \chead{\includegraphics[width=1cm]{../../template/4ourSquared_logo} } \rhead{Versione e Indice}
	\pagenumbering{Roman}
}

\newcommand{\makecontentsdetails}[1]{
	\clearpage
    \renewcommand{\footrulewidth}{0.4pt}
	\pagestyle{fancy}
	\lhead{4ourSquared} \chead{\includegraphics[width=1cm]{../../template/4ourSquared_logo}} \rhead{\nouppercase{\leftmark}}
	\pagenumbering{arabic}
    \lfoot{#1} \cfoot{} \rfoot{\thepage / \pageref{LastPage}}
}
\usepackage{graphicx}
\usepackage{hyperref}
\usepackage{makecell}

\newcommand{\settitolo}[1]{\newcommand{\titolo}{#1\\}}
\newcommand{\setprogetto}[1]{\newcommand{\progetto}{#1\\}}
\newcommand{\setcommittenti}[1]{\newcommand{\committenti}{#1\\}}
\newcommand{\setredattori}[1]{\newcommand{\redattori}{#1\\}}
\newcommand{\setrevisori}[1]{\newcommand{\revisori}{#1\\}}
\newcommand{\setresponsabili}[1]{\newcommand{\responsabili}{#1\\}}
\newcommand{\setversione}[1]{
	\ifdefined\versione\renewcommand{\versione}{#1\\}
	\else\newcommand{\versione}{#1\\}\fi
}
\newcommand{\setdestuso}[1]{\newcommand{\uso}{#1\\}}
\newcommand{\setdescrizione}[1]{\newcommand{\descrizione}{#1\\}}

\newcommand{\makefrontpage}{
	\begin{titlepage}
		\begin{center}

		\includegraphics[width=0.4\textwidth]{\templates/4ourSquared_logo}\\

		{\Large 4OURSQUARED}\\[6pt]
		\href{mailto://4oursquared.unipd@gmail.com}{4oursquared.unipd@gmail.com}\\
		
		\ifdefined\progetto
		\vspace{1cm}
		{\Large\progetto}
		{\large\committenti}
		\else\fi
		
		\vspace{1.5cm}
		{\LARGE\titolo}
		
		\vfill
		
		\begin{tabular}{r | l}
		\multicolumn{2}{c}{\textit{Informazioni}}\\
		\hline
		
		\ifdefined\redattori
			\textit{Redattori} &
			\makecell[l]{\redattori}\\
		\else\fi
		\ifdefined\revisori
			\textit{Revisori} &
			\makecell[l]{\revisori}\\
		\else\fi
		\ifdefined\responsabili
			\textit{Responsabili} &
			\makecell[l]{\responsabili}\\
		\else\fi
		
		\ifdefined\versione
			\textit{Versione} & \versione
		\else\fi
		
		\textit{Uso} & \uso
		
		\end{tabular}
		
		\vspace{2cm}
		
		\ifdefined\descrizione
		Descrizione
		\vspace{6pt}
		\hrule
		\descrizione
		\else\fi
		\end{center}
	\end{titlepage}
}
\usepackage{hyperref}
\usepackage{array}
\usepackage{tabularx}
\usepackage{adjustbox}

\newcounter{verscount}
\setcounter{verscount}{0}
\newcommand{\addversione}[5]{
	\ifdefined\setversione
		\setversione{#1}
	\else\fi
	\stepcounter{verscount}
	\expandafter\newcommand%
		\csname ver\theverscount \endcsname{#1&#2&#3&#4&#5}
}

\newcommand{\listversioni}{
	\ifnum\value{verscount}>1
		\csname ver\theverscount \endcsname
		\addtocounter{verscount}{-1}
		\\\hline
		\listversioni
	\else
		\csname ver\theverscount \endcsname\\\hline
	\fi
}

\newcommand{\makeversioni}{
	\begin{center}
		\begin{tabularx}{\textwidth}{|c|c|c|c|X|}
		\hline
		\textbf{Versione} & \textbf{Data} & \textbf{Redattore} & \textbf{Verificatore} & \textbf{Descrizione} \\
		\hline
		\listversioni
		\end{tabularx}
	\end{center}
	\clearpage
}

\settitolo{Motivazione delle Scelte}
\setprogetto{Lumos Minima}
\setcommittenti{Imola Informatica}
\setredattori{Ceccato Francesco \\ Soldà Matteo}
\setdestuso{esterno}
\setdescrizione{
Questo documento riporta i confronti tra i vari linguaggi presi in considerazione per lo sviluppo del progetto e la motivazione della scelta di uno rispetto agli altri.
}

\addversione{0.0.0}{09/05/2023}{Soldà Matteo}{Ceccato Francesco}{Prima stesura.}


\begin{document}
\makeindexdetails
\makefrontpage \makeversioni
\tableofcontents
\newpage
\clearpage
\makecontentsdetails{Motivazione delle Scelte}
\newpage
\section{Introduzione}
Il seguente documento ha come scopo principale quello di confrontare i vari linguaggi, database e framework con lo scopo di fornire un prodotto quanto più efficace, efficiente e aderente alle richieste del proponente.\\
\subsection{Sitografia}

\newpage
\section{Confronti}
\subsection{Database}
Come database, non sapendo al momento come saranno strutturati i dati che necessitano di essere salvati in maniera persistente, abbiamo preso in considerazione \href{https://mariadb.org/}{MariaDB} per i database SQL e \href{https://www.mongodb.com/}{MongoDB} per i database NoSQL. \\
Queste due scelte sono state fatte per due motivi principali:
\begin{itemize}
    \item Ricerca di un database computazionalmente leggero e che fosse aderente agli standard
    \item Ricerca di un database la cui community è attiva e la documentazione chiara e puntuale, così da poter risolvere quanto prima i problemi che si potessero presentare data la poca esperienza accumulata con i database nel tempo
\end{itemize}
\subsection{Backend}
Per quanto riguarda il backend, la scelta sarà vincolata dalla preferenza verso un linguaggio staticamente o dinamicamente tipizzato. \\
Qualora si scegliesse la tipizzazione statica, le scelte ricadono su:
\begin{itemize}
    \item \href{https://www.java.com/it/}{Java}
    \item \href{https://learn.microsoft.com/en-us/dotnet/csharp/}{C\#}
    \item \href{https://www.typescriptlang.org/}{TypeScript}
\end{itemize}
Mentre, scegliendo la tipizzazione dinamica, le scelte ricadrebbero su:
\begin{itemize}
    \item \href{https://www.python.org/}{Python}
    \item \href{https://www.javascript.com/}{JavaScript}
\end{itemize}
\subsubsection{C\#}
\subsubsection{Java}
\subsubsection{JavaScript}
\subsubsection{Python}
\subsubsection{Typescript}

\subsection{Backend Framework}
\subsubsection{C\#}
\begin{itemize}
    \item \href{https://dotnet.microsoft.com/en-us/apps/aspnet}{ASP.NET}
\end{itemize}
\subsubsection{Java}
\begin{itemize}
    \item \href{https://spring.io/}{Java Spring}
\end{itemize}
\subsubsection{JavaScript}
\begin{itemize}
    \item \href{https://nextjs.org/}{Next}
    \item \href{https://nodejs.org/}{Node}
\end{itemize}
\subsubsection{Python}
\begin{itemize}
    \item \href{https://www.djangoproject.com/}{Django}
    \item \href{https://flask.palletsprojects.com/}{Flask}
\end{itemize}
\subsubsection{Typescript}
\begin{itemize}
    \item \href{https://nestjs.com/}{Nest}
    \item \href{https://feathersjs.com/}{Feathers}
    \item \href{https://loopback.io/}{Loopback}
\end{itemize}
\subsection{Frontend}
Escludendo \textit{HTML5}, \textit{XHTML} e \textit{CSS} che saranno di sicuro usati per la parte di frontend della webapp, i linguaggi tra cui scegliere risultano essere solamente
\begin{itemize}
    \item \href{https://www.typescriptlang.org/}{TypeScript}
    \item \href{https://www.javascript.com/}{JavaScript}
\end{itemize}
\subsubsection{JavaScript}
\subsubsection{TypeScript}

\subsection{Frontend Framework}
\subsubsection{JavaScript}
\begin{itemize}
    \item \href{https://angularjs.org/}{Angular}
    \item \href{https://react.dev/}{React}
    \item \href{https://vuejs.org/}{Vue}
    \item \href{https://backbonejs.org/}{Backbone}
    \item \href{https://preactjs.com/}{Preact}
    \item \href{https://expressjs.com/}{Express}
\end{itemize}
\subsubsection{TypeScript}
\begin{itemize}
    \item \href{https://angularjs.org/}{Angular}
    \item \href{https://emberjs.com/}{Ember}
\end{itemize}

\newpage
\section{Scelte e Motivazioni}



\end{document} 

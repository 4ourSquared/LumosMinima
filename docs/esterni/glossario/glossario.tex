\documentclass[a4paper, 12pt]{article}

\newcommand{\templates}{../../template}
\usepackage[a4paper, margin=2.5cm]{geometry}
\usepackage{fancyvrb}
\usepackage{stackengine}
\usepackage{lastpage}

\usepackage{enumitem}
\setlist[itemize]{noitemsep}
\setlist[enumerate]{noitemsep}

\let\oldpar\paragraph
\renewcommand{\paragraph}[1]{\oldpar{#1\\}\noindent}

% Avoid dots in the table of contents, it mess with the gulpease calculation
\makeatletter
\renewcommand{\@dotsep}{10000} 
\makeatother

\newcommand{\makeindexdetails}{
	\pagestyle{fancy}
	\lhead{4ourSquared} \chead{\includegraphics[width=1cm]{../../template/4ourSquared_logo} } \rhead{Versione e Indice}
	\pagenumbering{Roman}
}

\newcommand{\makecontentsdetails}[1]{
	\clearpage
    \renewcommand{\footrulewidth}{0.4pt}
	\pagestyle{fancy}
	\lhead{4ourSquared} \chead{\includegraphics[width=1cm]{../../template/4ourSquared_logo}} \rhead{\nouppercase{\leftmark}}
	\pagenumbering{arabic}
    \lfoot{#1} \cfoot{} \rfoot{\thepage / \pageref{LastPage}}
}
\usepackage{graphicx}
\usepackage{hyperref}
\usepackage{makecell}

\newcommand{\settitolo}[1]{\newcommand{\titolo}{#1\\}}
\newcommand{\setprogetto}[1]{\newcommand{\progetto}{#1\\}}
\newcommand{\setcommittenti}[1]{\newcommand{\committenti}{#1\\}}
\newcommand{\setredattori}[1]{\newcommand{\redattori}{#1\\}}
\newcommand{\setrevisori}[1]{\newcommand{\revisori}{#1\\}}
\newcommand{\setresponsabili}[1]{\newcommand{\responsabili}{#1\\}}
\newcommand{\setversione}[1]{
	\ifdefined\versione\renewcommand{\versione}{#1\\}
	\else\newcommand{\versione}{#1\\}\fi
}
\newcommand{\setdestuso}[1]{\newcommand{\uso}{#1\\}}
\newcommand{\setdescrizione}[1]{\newcommand{\descrizione}{#1\\}}

\newcommand{\makefrontpage}{
	\begin{titlepage}
		\begin{center}

		\includegraphics[width=0.4\textwidth]{\templates/4ourSquared_logo}\\

		{\Large 4OURSQUARED}\\[6pt]
		\href{mailto://4oursquared.unipd@gmail.com}{4oursquared.unipd@gmail.com}\\
		
		\ifdefined\progetto
		\vspace{1cm}
		{\Large\progetto}
		{\large\committenti}
		\else\fi
		
		\vspace{1.5cm}
		{\LARGE\titolo}
		
		\vfill
		
		\begin{tabular}{r | l}
		\multicolumn{2}{c}{\textit{Informazioni}}\\
		\hline
		
		\ifdefined\redattori
			\textit{Redattori} &
			\makecell[l]{\redattori}\\
		\else\fi
		\ifdefined\revisori
			\textit{Revisori} &
			\makecell[l]{\revisori}\\
		\else\fi
		\ifdefined\responsabili
			\textit{Responsabili} &
			\makecell[l]{\responsabili}\\
		\else\fi
		
		\ifdefined\versione
			\textit{Versione} & \versione
		\else\fi
		
		\textit{Uso} & \uso
		
		\end{tabular}
		
		\vspace{2cm}
		
		\ifdefined\descrizione
		Descrizione
		\vspace{6pt}
		\hrule
		\descrizione
		\else\fi
		\end{center}
	\end{titlepage}
}
\usepackage{hyperref}
\usepackage{array}
\usepackage{tabularx}
\usepackage{adjustbox}

\newcounter{verscount}
\setcounter{verscount}{0}
\newcommand{\addversione}[5]{
	\ifdefined\setversione
		\setversione{#1}
	\else\fi
	\stepcounter{verscount}
	\expandafter\newcommand%
		\csname ver\theverscount \endcsname{#1&#2&#3&#4&#5}
}

\newcommand{\listversioni}{
	\ifnum\value{verscount}>1
		\csname ver\theverscount \endcsname
		\addtocounter{verscount}{-1}
		\\\hline
		\listversioni
	\else
		\csname ver\theverscount \endcsname\\\hline
	\fi
}

\newcommand{\makeversioni}{
	\begin{center}
		\begin{tabularx}{\textwidth}{|c|c|c|c|X|}
		\hline
		\textbf{Versione} & \textbf{Data} & \textbf{Redattore} & \textbf{Verificatore} & \textbf{Descrizione} \\
		\hline
		\listversioni
		\end{tabularx}
	\end{center}
	\clearpage
}

\settitolo{Glossario}
\setprogetto{Lumos Minima}
\setcommittenti{Imola Informatica}
\setredattori{Soldà Matteo}
\setdestuso{esterno}
\setdescrizione{
Questo documento riporta i termini specifici del dominio per i quali viene richiesta una definizione rigorosa e univoca.
}

\addversione{0.0.0}{24/03/2023}{Soldà Matteo}{Cavaliere Erica}{Prima stesura.}


\begin{document}
\makeindexdetails
\makefrontpage \makeversioni
\tableofcontents
\newpage
\clearpage
\makecontentsdetails{Glossario}

\section{B.}
\textbf{Blocco di Codice:} gruppo di istruzioni che devono essere eseguite in ordine. \\ \\

\newpage
\section{D.}
\textbf{Diagramma di Gantt:} strumento di supporto alla gestione dei progetti. Esso presenta l'arco temporale del progetto suddiviso in fasi incrementali e le mansioni che rappresentano il progetto stesso. \\

\newpage
\section{F.}
\textbf{Fornitura:} ciascuna delle serie di prodotti necessari al completamento, al funzionamento o all'efficienza di qualcosa.

\newpage
\section{I.}
\textbf{Implementazione:} realizzazione concreta di una procedura a partire dalla sua definizione logica. \\ \\
\textbf{Interfaccia:} dispositivo di collegamento con cui un software assicura la comunicazione tra due sistemi altrimenti incompatibili, oppure tra unità centrali e periferiche. \\ \\
\newpage
\section{P.}
\textbf{Piano di Progetto:} calendario di massima di un progetto riportante la stima dei costi di realizzazione, dei rischi attesi e della loro mitigazione e infine della suddivisione del lavoro in molteplici periodi successivi. Per ogni periodo devono essere riportati i seguenti elementi: obiettivi attesi, obiettivi raggiunti, rischi occorsi, conseguenze e mitigazione, costi osservati e aggiornamento del calendario futuro e della stima dei costi finali. \\ \\
\textbf{Piano di Qualifica:} specifica gli obiettivi quantitativi di qualità di prodotto e di processo, oltre a misurare il raggiungimento di tali obiettivi allo stato corrente pone le fondamenta per retrospettive e iniziative di auto-miglioramento. \\ \\
\textbf{Processo:} insieme di attività correlate e coese che trasformano bisogni (input) in prodotti (output) secondo regole date, utilizzando risorse nel farlo. \\ \\

\newpage
\section{V.}
\textbf{Versionamento:} tracciamento di un determinato stato di un software nel tempo. \\

\newpage
\section{W.}
\textbf{Way of Working:} rappresenta il come organizzare in maniera professionale le attività di progetto.\\ \\
\textbf{Web Application:} indica generalmente tutte le applicazioni fruibili tramite il web.


\end{document} 

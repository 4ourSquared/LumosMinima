\documentclass[a4paper, 12pt]{article}

\renewcommand{\contentsname}{Indice}
\newcommand{\templates}{../../template}
\usepackage[a4paper, margin=2.5cm]{geometry}
\usepackage{fancyvrb}
\usepackage{stackengine}
\usepackage{lastpage}

\usepackage{enumitem}
\setlist[itemize]{noitemsep}
\setlist[enumerate]{noitemsep}

\let\oldpar\paragraph
\renewcommand{\paragraph}[1]{\oldpar{#1\\}\noindent}

% Avoid dots in the table of contents, it mess with the gulpease calculation
\makeatletter
\renewcommand{\@dotsep}{10000} 
\makeatother

\newcommand{\makeindexdetails}{
	\pagestyle{fancy}
	\lhead{4ourSquared} \chead{\includegraphics[width=1cm]{../../template/4ourSquared_logo} } \rhead{Versione e Indice}
	\pagenumbering{Roman}
}

\newcommand{\makecontentsdetails}[1]{
	\clearpage
    \renewcommand{\footrulewidth}{0.4pt}
	\pagestyle{fancy}
	\lhead{4ourSquared} \chead{\includegraphics[width=1cm]{../../template/4ourSquared_logo}} \rhead{\nouppercase{\leftmark}}
	\pagenumbering{arabic}
    \lfoot{#1} \cfoot{} \rfoot{\thepage / \pageref{LastPage}}
}
\usepackage{graphicx}
\usepackage{hyperref}
\usepackage{makecell}

\newcommand{\settitolo}[1]{\newcommand{\titolo}{#1\\}}
\newcommand{\setprogetto}[1]{\newcommand{\progetto}{#1\\}}
\newcommand{\setcommittenti}[1]{\newcommand{\committenti}{#1\\}}
\newcommand{\setredattori}[1]{\newcommand{\redattori}{#1\\}}
\newcommand{\setrevisori}[1]{\newcommand{\revisori}{#1\\}}
\newcommand{\setresponsabili}[1]{\newcommand{\responsabili}{#1\\}}
\newcommand{\setversione}[1]{
	\ifdefined\versione\renewcommand{\versione}{#1\\}
	\else\newcommand{\versione}{#1\\}\fi
}
\newcommand{\setdestuso}[1]{\newcommand{\uso}{#1\\}}
\newcommand{\setdescrizione}[1]{\newcommand{\descrizione}{#1\\}}

\newcommand{\makefrontpage}{
	\begin{titlepage}
		\begin{center}

		\includegraphics[width=0.4\textwidth]{\templates/4ourSquared_logo}\\

		{\Large 4OURSQUARED}\\[6pt]
		\href{mailto://4oursquared.unipd@gmail.com}{4oursquared.unipd@gmail.com}\\
		
		\ifdefined\progetto
		\vspace{1cm}
		{\Large\progetto}
		{\large\committenti}
		\else\fi
		
		\vspace{1.5cm}
		{\LARGE\titolo}
		
		\vfill
		
		\begin{tabular}{r | l}
		\multicolumn{2}{c}{\textit{Informazioni}}\\
		\hline
		
		\ifdefined\redattori
			\textit{Redattori} &
			\makecell[l]{\redattori}\\
		\else\fi
		\ifdefined\revisori
			\textit{Revisori} &
			\makecell[l]{\revisori}\\
		\else\fi
		\ifdefined\responsabili
			\textit{Responsabili} &
			\makecell[l]{\responsabili}\\
		\else\fi
		
		\ifdefined\versione
			\textit{Versione} & \versione
		\else\fi
		
		\textit{Uso} & \uso
		
		\end{tabular}
		
		\vspace{2cm}
		
		\ifdefined\descrizione
		Descrizione
		\vspace{6pt}
		\hrule
		\descrizione
		\else\fi
		\end{center}
	\end{titlepage}
}
\usepackage{hyperref}
\usepackage{array}
\usepackage{tabularx}
\usepackage{adjustbox}

\newcounter{verscount}
\setcounter{verscount}{0}
\newcommand{\addversione}[5]{
	\ifdefined\setversione
		\setversione{#1}
	\else\fi
	\stepcounter{verscount}
	\expandafter\newcommand%
		\csname ver\theverscount \endcsname{#1&#2&#3&#4&#5}
}

\newcommand{\listversioni}{
	\ifnum\value{verscount}>1
		\csname ver\theverscount \endcsname
		\addtocounter{verscount}{-1}
		\\\hline
		\listversioni
	\else
		\csname ver\theverscount \endcsname\\\hline
	\fi
}

\newcommand{\makeversioni}{
	\begin{center}
		\begin{tabularx}{\textwidth}{|c|c|c|c|X|}
		\hline
		\textbf{Versione} & \textbf{Data} & \textbf{Redattore} & \textbf{Verificatore} & \textbf{Descrizione} \\
		\hline
		\listversioni
		\end{tabularx}
	\end{center}
	\clearpage
}

\settitolo{Glossario}
\setprogetto{Lumos Minima}
\setcommittenti{Imola Informatica}
\setredattori{Alberti Nicolas \\ Cavaliere Erica \\ Soldà Matteo \\ Salami Lorenzo}
\setdestuso{esterno}
\setdescrizione{
Questo documento riporta i termini specifici del dominio per i quali viene richiesta una definizione rigorosa e univoca.
}
\addversione{0.0.0}{24/03/2023}{Soldà Matteo}{Cavaliere Erica}{Prima stesura.}
\addversione{0.0.1}{03/05/2023}{Nicolas Alberti}{Soldà Matteo}{Aggiunta parole al glossario, lettera T.}
\addversione{0.0.2}{10/05/2023}{Soldà Matteo}{Alberti Nicolas}{Aggiunti nuovi vocaboli derivanti da \textit{Motivazione delle Scelte} e anche da \textit{Norme di Progetto}}
\addversione{0.0.3}{30/05/2023}{Cavaliere Erica}{Alberti Nicolas}{Aggiunti nuovi vocaboli derivanti dall'\textit{Analisi dei requisiti}}
\addversione{0.0.4}{09/06/2023}{Salami Lorenzo}{Ceccato Francesco}{Aggiunti nuovi vocaboli derivanti dall'\textit{Analisi dei requisiti}}
\addversione{0.1.0}{20/07/2023}{Soldà Matteo}{Alberti Nicolas}{Verifica del documento per RTB}
\addversione{1.0.0}{20/07/2023}{Soldà Matteo}{Alberti Nicolas}{Approvazione del documento per candidatura RTB}
\addversione{1.0.1}{15/08/2023}{Alberti Nicolas}{Soldà Matteo}{Modificato l'indice del glossario inserendo le lettere al posto dei numeri.}
\addversione{1.0.2}{22/08/2023}{Soldà Matteo}{Cavaliere Erica}{Aggiunta parola al glossatio, lettera J.}
\addversione{1.0.3}{24/08/2023}{Soldà Matteo}{Alberti Nicolas}{Aggiunte parole al glossario, lettera P. e T. }
\addversione{1.1.0}{26/09/2023}{Alberti Nicolas}{Soldà Matteo}{Verifica per
\textit{PB}}
\addversione{2.0.0}{26/09/2023}{Soldà Matteo}{Alberti Nicolas}{Approvazione per candidatura \textit{PB}}


\begin{document}
\makeindexdetails
\makefrontpage \makeversioni


\tableofcontents

\newpage
\clearpage
\makecontentsdetails{Glossario}

\section*{A.}
\markboth{A.}{}
\addcontentsline{toc}{section}{A.}
\textbf{Area illuminata:} un luogo composto da un insieme di un numero arbitrario di \textit{impianti luminosi\textsubscript{G}} e sensori.

\newpage
\section*{B.}
\markboth{B.}{}
\addcontentsline{toc}{section}{B.}
\textbf{Back-end:} parte di un sistema software che elabora i dati generati dal \textit{front-end\textsubscript{G}}. \\ \\
\textbf{Blocco di Codice:} gruppo di istruzioni che devono essere eseguite in ordine. \\ \\
\textbf{Branch:} in \textit{GitHub\textsubscript{G}} è un ambiente isolato della
\textit{repository\textsubscript{G}} che permette di duplicare il codice
sorgente e modificarne alcune parti, apportando modifiche che non intaccano il codice principale del progetto. \\ \\

\newpage
\section*{D.}
\markboth{D.}{}
\addcontentsline{toc}{section}{D.}
\textbf{Dashboard:} pannello di visualizzazione dei dati che fornisce una panoramica in tempo reale delle prestazioni e delle metriche di un progetto, servizio o specifica risorsa. \\ \\
\textbf{Diagramma di Gantt:} strumento di supporto alla gestione dei progetti. Esso presenta l'arco temporale del progetto suddiviso in fasi incrementali e le mansioni che rappresentano il progetto stesso. \\

\newpage
\section*{F.}
\markboth{F.}{}
\addcontentsline{toc}{section}{F.}
\textbf{Fornitura:} ciascuna delle serie di prodotti necessari al completamento, al funzionamento o all'efficienza di qualcosa. \\ \\
\textbf{Front-end:} parte di un sistema software che gestisce l'interazione con l'utente o con sistemi esterni. \\ \\
\textbf{Full-stack:} insieme di un sistema o di applicazione informatica, comprendente sia il frontend che il backend.\\ \\

\newpage
\section*{G.}
\markboth{G.}{}
\addcontentsline{toc}{section}{G.}
\textbf{Git:} sistema di controllo di versione distribuito, focalizzato
sull'agevolare il coordinamento dello sviluppo di codice tra più persone. \\ \\
\textbf{GitHub:} servizio di hosting per progetti software per
\textit{repository\textsubscript{G}} \textit{Git\textsubscript{G}}. Permette di
utilizzare software aggiuntivi per la collaborazione, quali:
\begin{itemize}
    \item sistema di \textit{ticketing\textsubscript{G}};
    \item sistema per la gestione delle attività di progetto;
    \item sistema per l'integraziopne continua.
\end{itemize}
\newpage

\section*{I.}
\markboth{I.}{}
\addcontentsline{toc}{section}{I.}
\textbf{IoT}: \textit{Internet of Things}, descrive la rete di oggetti fisici che hanno sensori, software e altre tecnologie integrate allo scopo di connettere e scambiare dati con altri dispositivi e sistemi tramite internet. \\ \\
\textbf{Impianto luminoso:} elemento composto da una fonte luminosa. \\ \\
\textbf{Implementazione:} realizzazione concreta di una procedura a partire dalla sua definizione logica. \\ \\
\textbf{Intensità luminosa:} quantità di luce prodotta da un \textit{impianto luminoso\textsubscript{G}}. \\ \\
\textbf{Interfaccia:} dispositivo di collegamento con cui un software assicura la comunicazione tra due sistemi altrimenti incompatibili, oppure tra unità centrali e periferiche. \\ \\

\newpage
\section*{J.}
\markboth{J.}{}
\addcontentsline{toc}{section}{J.}
\textbf{JWT - JSON Web Token:} standard per la creazione di dati dove la firma e la cifratura sono opzionali. Il payload di questo token può essere cifrato utilizzando una chiave private e/o una chiave pubblica.

\newpage
\section*{L.}
\markboth{L.}{}
\addcontentsline{toc}{section}{L.}
\textbf{Landing Page:}: pagina web che ha il compito di accogliere l'utente che vi arriva, fornendogli informazioni e aiutandolo a trovare ciò che cerca. \\ \\

\newpage
\section*{M.}
\markboth{M.}{}
\addcontentsline{toc}{section}{M.}
\textbf{Multithreading:} tecnicna utilizzata nella programmazione per consentire a un singolo processo di avere più \textit{thread\textsubscript{G}} di esecuzione. Ogni thread può funzionare in modo indipendente e parallelo, consentendo un uso più efficiente delle risorse. Ciò può aiutare a migliorare le prestazioni complessive di un programma e renderlo più sensibile all'input di un utente. \\ \\

\newpage
\section*{P.}
\markboth{P.}{}
\addcontentsline{toc}{section}{P.}
\textbf{Parsing:} processo di analisi di un flsso di dati in ingresso per determinare la correttenza della sua struttura ad una data grammatica formale. \\ \\
\textbf{Piano di Progetto:} calendario di massima di un progetto riportante la stima dei costi di realizzazione, dei rischi attesi e della loro mitigazione e infine della suddivisione del lavoro in molteplici periodi successivi. Per ogni periodo devono essere riportati i seguenti elementi: obiettivi attesi, obiettivi raggiunti, rischi occorsi, conseguenze e mitigazione, costi osservati e aggiornamento del calendario futuro e della stima dei costi finali. \\ \\
\textbf{Piano di Qualifica:} specifica gli obiettivi quantitativi di qualità di prodotto e di processo, oltre a misurare il raggiungimento di tali obiettivi allo stato corrente pone le fondamenta per retrospettive e iniziative di auto-miglioramento. \\ \\
\textbf{Polling Time:} indica il tempo che intercorre tra una operazione e l'altra che si svolgono in loop. Essa può essere seguita da una lettura o scrittura. \\ \\
\textbf{Processo:} insieme di attività correlate e coese che trasformano bisogni (input) in prodotti (output) secondo regole date, utilizzando risorse nel farlo. \\ \\
\textbf{Push/Pull:} In ambito web application, il push si riferisce alla comunicazione in cui il server invia attivamente i dati o gli aggiornamenti al client senza che vengano richiesti esplicitamente. Il pull, invece, implica che il client richieda esplicitamente i dati o gli aggiornamenti al server quando ne ha bisogno. Il push è utilizzato per aggiornamenti in tempo reale, mentre il pull viene utilizzato per richieste specifiche del client. \\ \\

\newpage
\section*{R.}
\markboth{R.}{}
\addcontentsline{toc}{section}{R.}
\textbf{Repository:} è una cartella remota che permette il salvataggio dei file
di progetto ed utilizza un sistema di controllo di versione. Quando
viene nominato, ci si riferisce al repository \textit{GitHub\textsubscript{G}} reperibile all’indirizzo \texttt{https://github.com/4ourSquared/LumosMinima}.  \\ \\
\textbf{Responsive:}Detto di sito web in grado di adattarsi graficamente in maniera automatica a qualsiasi dispositivo su cui viene visualizzato. \\ \\

\newpage
\section*{S.}
\markboth{S.}{}
\addcontentsline{toc}{section}{S.}
\textbf{Soglia:} una fascia di tolleranza utilizzata per determinare se l'\textit{intensità luminosa\textsubscript{G}} è al di sopra o al di sotto di un limite in una determinata fascia oraria. \\ \\

\newpage
\section*{T.}
\markboth{T.}{}
\addcontentsline{toc}{section}{T.}
\textbf{Thread:} suddivisione di un processo in più istanze o sottoprocessi che vengono eseguiti concorrentemente da un sistema di elaborazione. \\ \\
\textbf{Ticket:} è l'astrazione presente in GitHub Issues con cui si rappresenta
un'attività di progetto: le attività sono solitamente atomiche e completabili in
un periodo di tempo limitato. Ad ogni ticket si può associare o creare un branch
nella repository del progetto.\\ \\
% TODO: Controllare se nelle Norme di Progetto sono presenti le parole "Branch" e "Repository".
\textbf{Ticketing:} è il metodo con cui si rappresentano tutte le attività di
progetto: è composto di più \textit{ticket\textsubscript{G}}, che permettono una visualizzazione rapida ed
efficace di ogni incarico effettuato dai membri del gruppo. \\ \\
\textbf{Token:} blocco di testo categorizzato da lessemi e che richiede il \textit{parsing\textsubscript{G}}
\newpage
\section*{V.}
\markboth{V.}{}
\addcontentsline{toc}{section}{V.}
\textbf{Versionamento:} tracciamento di un determinato stato di un software nel tempo. \\ \\

\newpage
\section*{W.}
\markboth{W.}{}
\addcontentsline{toc}{section}{W.}
\textbf{Way of Working:} rappresenta il come organizzare in maniera professionale le attività di progetto.\\ \\
\textbf{Web Application:} indica generalmente tutte le applicazioni fruibili tramite il web.





\end{document}

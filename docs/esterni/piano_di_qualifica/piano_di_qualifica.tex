\documentclass[a4paper, 12pt]{article}
\usepackage{amsmath}
\usepackage{pgfplots}
\usepackage{tabularx}
\usepgfplotslibrary{dateplot}

\newcommand{\templates}{../../template}
\usepackage[a4paper, margin=2.5cm]{geometry}
\usepackage{fancyvrb}
\usepackage{stackengine}
\usepackage{lastpage}

\usepackage{enumitem}
\setlist[itemize]{noitemsep}
\setlist[enumerate]{noitemsep}

\let\oldpar\paragraph
\renewcommand{\paragraph}[1]{\oldpar{#1\\}\noindent}

% Avoid dots in the table of contents, it mess with the gulpease calculation
\makeatletter
\renewcommand{\@dotsep}{10000} 
\makeatother

\newcommand{\makeindexdetails}{
	\pagestyle{fancy}
	\lhead{4ourSquared} \chead{\includegraphics[width=1cm]{../../template/4ourSquared_logo} } \rhead{Versione e Indice}
	\pagenumbering{Roman}
}

\newcommand{\makecontentsdetails}[1]{
	\clearpage
    \renewcommand{\footrulewidth}{0.4pt}
	\pagestyle{fancy}
	\lhead{4ourSquared} \chead{\includegraphics[width=1cm]{../../template/4ourSquared_logo}} \rhead{\nouppercase{\leftmark}}
	\pagenumbering{arabic}
    \lfoot{#1} \cfoot{} \rfoot{\thepage / \pageref{LastPage}}
}
\usepackage{graphicx}
\usepackage{hyperref}
\usepackage{makecell}

\newcommand{\settitolo}[1]{\newcommand{\titolo}{#1\\}}
\newcommand{\setprogetto}[1]{\newcommand{\progetto}{#1\\}}
\newcommand{\setcommittenti}[1]{\newcommand{\committenti}{#1\\}}
\newcommand{\setredattori}[1]{\newcommand{\redattori}{#1\\}}
\newcommand{\setrevisori}[1]{\newcommand{\revisori}{#1\\}}
\newcommand{\setresponsabili}[1]{\newcommand{\responsabili}{#1\\}}
\newcommand{\setversione}[1]{
	\ifdefined\versione\renewcommand{\versione}{#1\\}
	\else\newcommand{\versione}{#1\\}\fi
}
\newcommand{\setdestuso}[1]{\newcommand{\uso}{#1\\}}
\newcommand{\setdescrizione}[1]{\newcommand{\descrizione}{#1\\}}

\newcommand{\makefrontpage}{
	\begin{titlepage}
		\begin{center}

		\includegraphics[width=0.4\textwidth]{\templates/4ourSquared_logo}\\

		{\Large 4OURSQUARED}\\[6pt]
		\href{mailto://4oursquared.unipd@gmail.com}{4oursquared.unipd@gmail.com}\\
		
		\ifdefined\progetto
		\vspace{1cm}
		{\Large\progetto}
		{\large\committenti}
		\else\fi
		
		\vspace{1.5cm}
		{\LARGE\titolo}
		
		\vfill
		
		\begin{tabular}{r | l}
		\multicolumn{2}{c}{\textit{Informazioni}}\\
		\hline
		
		\ifdefined\redattori
			\textit{Redattori} &
			\makecell[l]{\redattori}\\
		\else\fi
		\ifdefined\revisori
			\textit{Revisori} &
			\makecell[l]{\revisori}\\
		\else\fi
		\ifdefined\responsabili
			\textit{Responsabili} &
			\makecell[l]{\responsabili}\\
		\else\fi
		
		\ifdefined\versione
			\textit{Versione} & \versione
		\else\fi
		
		\textit{Uso} & \uso
		
		\end{tabular}
		
		\vspace{2cm}
		
		\ifdefined\descrizione
		Descrizione
		\vspace{6pt}
		\hrule
		\descrizione
		\else\fi
		\end{center}
	\end{titlepage}
}
\usepackage{hyperref}
\usepackage{array}
\usepackage{tabularx}
\usepackage{adjustbox}

\newcounter{verscount}
\setcounter{verscount}{0}
\newcommand{\addversione}[5]{
	\ifdefined\setversione
		\setversione{#1}
	\else\fi
	\stepcounter{verscount}
	\expandafter\newcommand%
		\csname ver\theverscount \endcsname{#1&#2&#3&#4&#5}
}

\newcommand{\listversioni}{
	\ifnum\value{verscount}>1
		\csname ver\theverscount \endcsname
		\addtocounter{verscount}{-1}
		\\\hline
		\listversioni
	\else
		\csname ver\theverscount \endcsname\\\hline
	\fi
}

\newcommand{\makeversioni}{
	\begin{center}
		\begin{tabularx}{\textwidth}{|c|c|c|c|X|}
		\hline
		\textbf{Versione} & \textbf{Data} & \textbf{Redattore} & \textbf{Verificatore} & \textbf{Descrizione} \\
		\hline
		\listversioni
		\end{tabularx}
	\end{center}
	\clearpage
}

\settitolo{Piano di Qualifica}
\setredattori{Lorenzo Salami}
\setdestuso{esterno}
\setdescrizione{
Questo documento serve a definire le metriche e i criteri di accettazione dei prodotti.
}


\addversione{0.0.0}{18/04/2023}{Salami Lorenzo}{Soldà Matteo}{Stesura iniziale.}
\addversione{0.0.1}{24/04/2023}{Soldà Matteo}{Salami Lorenzo}{Aggiunta delle intestazioni e dei piè di pagina}


\begin{document}

\makefrontpage
\makeindexdetails
\makeversioni
\tableofcontents
\clearpage
\makecontentsdetails

\section{Qualità di prodotto}

\subsection{Documentazione}
\subsubsection{Indice di Gulpease}
\[ \text{Indice di Gulpease} = 89 + \frac{300*\text{\#frasi} - 10*\text{\#lettere}}{\text{\#parole}} \]
\begin{itemize}
	\item \#lettere: numero di caratteri alfanumerici;
	\item \#parole: numero di gruppi di caratteri alfanumerici;
	\item \#frasi: numero di gruppi di punti o punti e virgola consecutivi.
\end{itemize}

\subparagraph{Prodotti coinvolti:}
\begin{center}
	\begin{tabularx}{\textwidth}{|X|X|X|}
		\hline
		\textbf{Prodotto} & \textbf{Valore accettabile} & \textbf{Valore ottimale } \\
		\hline
		Documenti interni & $>$ 40                      & $>$ 60                    \\
		\hline
		Documenti esterni & $>$ 50                      & $>$ 60                    \\
		\hline
	\end{tabularx}\\[8pt]
	\mbox{}\\
\end{center}

\subparagraph{Riferimenti:} \underline{\href{http://www.corrige.it/leggibilita/lindice-gulpease/}{http://www.corrige.it/leggibilita/lindice-gulpease/}}

\subsection{Prodotti software}

\subsubsection{Copertura statement}
La metrica si basa sullo statement coverage.

\subparagraph{Prodotti coinvolti:}
\begin{center}
	\begin{tabularx}{\textwidth}{|X|X|X|}
		\hline
		\textbf{Prodotto} & \textbf{Valore accettabile } & \textbf{Valore ottimale } \\
		\hline
		Software          & $>$ 80\%                     & $>$ 95\%                     \\
		\hline
	\end{tabularx}\\[8pt]
	\mbox{}\\
\end{center}
\subsubsection{Copertura branch}
La metrica si basa sul branch coverage.

\subparagraph{Prodotti coinvolti:}
\begin{center}
	\begin{tabularx}{\textwidth}{|X|X|X|}
		\hline
		\textbf{Prodotto} & \textbf{Valore accettabile } & \textbf{Valore ottimale } \\
		\hline
		Software          & $>$ 80\%                     & $>$ 95\%                     \\
		\hline
	\end{tabularx}\\[8pt]
	\mbox{}\\
\end{center}

\newpage
\section{Qualità di processo}
\subsubsection{Time variance}
La metrica si basa sulla variazione percentuale rispetto alla stima iniziale.

\subparagraph{Prodotti coinvolti:}
\begin{center}
	\begin{tabularx}{\textwidth}{|X|X|X|}
		\hline
		\textbf{Prodotto} & \textbf{Valore accettabile } & \textbf{Valore ottimale } \\
		\hline
		Software          & $<$ 20\%                     & 0\%                       \\
		\hline
		Documentazione    & $<$ 20\%                     & 0\%                       \\
		\hline
	\end{tabularx}\\[8pt]
	\mbox{}\\
\end{center}
\subsubsection{Budget variance}
La metrica si basa sulla variazione percentuale rispetto alla stima iniziale.

\subparagraph{Prodotti coinvolti:}
\begin{center}
	\begin{tabularx}{\textwidth}{|X|X|X|}
		\hline
		\textbf{Prodotto} & \textbf{Valore accettabile } & \textbf{Valore ottimale } \\
		\hline
		Software          & $<$ 20\%                     & 0\%                       \\
		\hline
		Documentazione    & $<$ 20\%                     & 0\%                       \\
		\hline
	\end{tabularx}\\[8pt]
	\mbox{}\\
\end{center}

\end{document}
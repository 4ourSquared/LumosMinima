\documentclass[a4paper, 12pt]{article}

\newcommand{\templates}{../../template}
\usepackage[a4paper, margin=2.5cm]{geometry}
\usepackage{fancyvrb}
\usepackage{stackengine}
\usepackage{lastpage}

\usepackage{enumitem}
\setlist[itemize]{noitemsep}
\setlist[enumerate]{noitemsep}

\let\oldpar\paragraph
\renewcommand{\paragraph}[1]{\oldpar{#1\\}\noindent}

% Avoid dots in the table of contents, it mess with the gulpease calculation
\makeatletter
\renewcommand{\@dotsep}{10000} 
\makeatother

\newcommand{\makeindexdetails}{
	\pagestyle{fancy}
	\lhead{4ourSquared} \chead{\includegraphics[width=1cm]{../../template/4ourSquared_logo} } \rhead{Versione e Indice}
	\pagenumbering{Roman}
}

\newcommand{\makecontentsdetails}[1]{
	\clearpage
    \renewcommand{\footrulewidth}{0.4pt}
	\pagestyle{fancy}
	\lhead{4ourSquared} \chead{\includegraphics[width=1cm]{../../template/4ourSquared_logo}} \rhead{\nouppercase{\leftmark}}
	\pagenumbering{arabic}
    \lfoot{#1} \cfoot{} \rfoot{\thepage / \pageref{LastPage}}
}
\usepackage{graphicx}
\usepackage{hyperref}
\usepackage{makecell}

\newcommand{\settitolo}[1]{\newcommand{\titolo}{#1\\}}
\newcommand{\setprogetto}[1]{\newcommand{\progetto}{#1\\}}
\newcommand{\setcommittenti}[1]{\newcommand{\committenti}{#1\\}}
\newcommand{\setredattori}[1]{\newcommand{\redattori}{#1\\}}
\newcommand{\setrevisori}[1]{\newcommand{\revisori}{#1\\}}
\newcommand{\setresponsabili}[1]{\newcommand{\responsabili}{#1\\}}
\newcommand{\setversione}[1]{
	\ifdefined\versione\renewcommand{\versione}{#1\\}
	\else\newcommand{\versione}{#1\\}\fi
}
\newcommand{\setdestuso}[1]{\newcommand{\uso}{#1\\}}
\newcommand{\setdescrizione}[1]{\newcommand{\descrizione}{#1\\}}

\newcommand{\makefrontpage}{
	\begin{titlepage}
		\begin{center}

		\includegraphics[width=0.4\textwidth]{\templates/4ourSquared_logo}\\

		{\Large 4OURSQUARED}\\[6pt]
		\href{mailto://4oursquared.unipd@gmail.com}{4oursquared.unipd@gmail.com}\\
		
		\ifdefined\progetto
		\vspace{1cm}
		{\Large\progetto}
		{\large\committenti}
		\else\fi
		
		\vspace{1.5cm}
		{\LARGE\titolo}
		
		\vfill
		
		\begin{tabular}{r | l}
		\multicolumn{2}{c}{\textit{Informazioni}}\\
		\hline
		
		\ifdefined\redattori
			\textit{Redattori} &
			\makecell[l]{\redattori}\\
		\else\fi
		\ifdefined\revisori
			\textit{Revisori} &
			\makecell[l]{\revisori}\\
		\else\fi
		\ifdefined\responsabili
			\textit{Responsabili} &
			\makecell[l]{\responsabili}\\
		\else\fi
		
		\ifdefined\versione
			\textit{Versione} & \versione
		\else\fi
		
		\textit{Uso} & \uso
		
		\end{tabular}
		
		\vspace{2cm}
		
		\ifdefined\descrizione
		Descrizione
		\vspace{6pt}
		\hrule
		\descrizione
		\else\fi
		\end{center}
	\end{titlepage}
}
\usepackage{hyperref}
\usepackage{array}
\usepackage{tabularx}
\usepackage{adjustbox}

\newcounter{verscount}
\setcounter{verscount}{0}
\newcommand{\addversione}[5]{
	\ifdefined\setversione
		\setversione{#1}
	\else\fi
	\stepcounter{verscount}
	\expandafter\newcommand%
		\csname ver\theverscount \endcsname{#1&#2&#3&#4&#5}
}

\newcommand{\listversioni}{
	\ifnum\value{verscount}>1
		\csname ver\theverscount \endcsname
		\addtocounter{verscount}{-1}
		\\\hline
		\listversioni
	\else
		\csname ver\theverscount \endcsname\\\hline
	\fi
}

\newcommand{\makeversioni}{
	\begin{center}
		\begin{tabularx}{\textwidth}{|c|c|c|c|X|}
		\hline
		\textbf{Versione} & \textbf{Data} & \textbf{Redattore} & \textbf{Verificatore} & \textbf{Descrizione} \\
		\hline
		\listversioni
		\end{tabularx}
	\end{center}
	\clearpage
}

\settitolo{Norme di Progetto}
\setprogetto{Lumos Minima}
\setcommittenti{Imola Informatica}
\setredattori{Soldà Matteo}
\setdestuso{interno}
\setdescrizione{
Questo documento contiene le procedure, gli strumenti e i criteri di qualità che verranno usati nel progetto.
}

\addversione{0.0.0}{24/03/2023}{Soldà Matteo}{Brotto Romina}{Prima stesura.}

\begin{document}
\makefrontpage \makeversioni
\tableofcontents
\newpage

\section{Introduzione}
\subsection{Scopo del Documento}
Il documento si prefigge lo scopo di definire i metodi e le attività chiave legate al \textit{Way of Working}. Tutti i membri del gruppo si impegnano a seguire quanto riportato.

\section{Processi Primari}
\subsection{Fornitura}
\subsubsection{Repository Pubblica}
Durante lo sviluppo del progetto, il codice e la documentazione saranno depositati in una repository pubblica. La parte di interesse del proponente e del committente riguarda la cartella \textit{public}. \\
La cartella sopra citata contiene la documentazione necessaria alle revisioni di periodo con il committente. Tale cartella rispecchia la struttura della cartella \textit{docs}, situata nella parte privata. \\
I file che prevedono versionamento, riporteranno al loro interno la numerazione relativa alla versione attuale e un riassunto delle versioni precedenti. \\ 
Ogni modifica della parte pubblica deve essere approvata, tramite \textit{pull request}, dal responsabile.

\subsubsection{Contatti}
Tutti i contatti ufficiali, sia con il proponente che con il committente, avverranno tramite la mail del gruppo \href{mailto:4oursquared.unipd@gmail.com}{4oursquared.unipd@gmail.com}.\\
Le mail per la presentazione ad una revisione di periodo necessita di una conferma unanime, valutata la preparazione. Questa mail deve contenere la lettera di presentazione con i link ai vari documenti aggiunti/aggiornati nella parte pubblica della repository.

\subsection{Sviluppo}
\subsubsection{Organizzazione dei File}
Tutta la documentazione sarà contenuta all'interno della cartella \textit{docs/}. Questa sarà inoltre suddivisa in \textit{interna} ed \textit{esterna}, le quali conterranno rispettivamente la documentazione interna al gruppo e quella da condividere con il proponente e con il committente. \\
I verbali interni saranno conservati nella sottocartella \textit{docs/interni/verbali} mentre quelli esterni saranno conservati in \textit{docs/esterni/verbali}.\\
Per ogni documento esisterà una nuova sottocartella omonima.

\section{Processi Secondari}
\subsection{Documentazione}
\subsubsection{Ciclo di Vita}
Ogni documento seguirà il seguente flusso:
\begin{itemize}
    \item \textbf{Pianificazione} \\ Il documento viene progettato ad alto livello secondo quanto richiesto.
    \item \textbf{Redazione} \\ Un membro del gruppo, nel ruolo di redattore, stila il documento.
    \item \textbf{Revisione} \\ Un membro del gruppo, nel ruolo di revisore, controlla l'assenza di errori grammaticali e il contenuto sia conforme alle norme di progetto.
    \item \textbf{Approvazione} \\ Un membro del gruppo, nel ruolo di responsabile, verifica che il contenuto del documento sia corretto.
\end{itemize}

\subsubsection{Strumenti Utilizzati}
Per la stesura dei documenti verrà utilizzato il linguaggio \LaTeX. \\
Affinché i documenti risultino stilisticamente coerenti, dovranno includere al loro interno il file \textit{docs/template/style.tex} per lo stile generale e il file \textit{docs/template/front\_page.tex} per utilizzare il template della prima pagina.

\subsubsection{Struttura}
Ogni documento sarà così strutturato:
\begin{enumerate}
    \item Pagina di intestazione
    \item Pagina con lista delle versioni, ove necessario
    \item Pagina indice, ove necessario
    \item Una o più pagine di contenuto
\end{enumerate}

Le varie sezioni saranno così definite:

\textbf{Pagina di Intestazione} \\
\begin{itemize}
    \item Logo e nome del gruppo
    \item Nome del progetto e azienda cliente
    \item Mail di contatto
    \item Titolo del documento
    \item Info generali
    \begin{itemize}
        \item Redattori
        \item Revisori
        \item Responsabile
        \item Versione, ove necessario
        \item Destinazione d'uso (interno o esterno)
    \end{itemize}
    \item Sinossi
\end{itemize}

\textbf{Elenco delle Versioni} \\
L'elenco delle versioni è una tabella così definita:
\begin{itemize}
    \item Versione
    \item Data
    \item Redattore
    \item Revisore
    \item Descrizione
\end{itemize}
Le righe della tabella sono organizzate in ordine cronologico inverso.

\textbf{Indice} \\
L'indice è stato creato utilizzando il comando \textit{\textbackslash tableofcontents}

\subsection{Versionamento}
\subsubsection{Repository Github}
Per il versionamento del progetto si è deciso di utilizzare \textit{Git} sulla piattaforma \textit{GitHub}. La repository è raggiungibile tramite l'indirizzo \href{https://github.com/4ourSquared/LumosMinima}{https://github.com/4ourSquared/LumosMinima}.

\end{document}
\documentclass[a4paper, 12pt]{article}

\newcommand{\templates}{../../template}
\usepackage[a4paper, margin=2.5cm]{geometry}
\usepackage{fancyvrb}
\usepackage{stackengine}
\usepackage{lastpage}

\usepackage{enumitem}
\setlist[itemize]{noitemsep}
\setlist[enumerate]{noitemsep}

\let\oldpar\paragraph
\renewcommand{\paragraph}[1]{\oldpar{#1\\}\noindent}

% Avoid dots in the table of contents, it mess with the gulpease calculation
\makeatletter
\renewcommand{\@dotsep}{10000} 
\makeatother

\newcommand{\makeindexdetails}{
	\pagestyle{fancy}
	\lhead{4ourSquared} \chead{\includegraphics[width=1cm]{../../template/4ourSquared_logo} } \rhead{Versione e Indice}
	\pagenumbering{Roman}
}

\newcommand{\makecontentsdetails}[1]{
	\clearpage
    \renewcommand{\footrulewidth}{0.4pt}
	\pagestyle{fancy}
	\lhead{4ourSquared} \chead{\includegraphics[width=1cm]{../../template/4ourSquared_logo}} \rhead{\nouppercase{\leftmark}}
	\pagenumbering{arabic}
    \lfoot{#1} \cfoot{} \rfoot{\thepage / \pageref{LastPage}}
}
\usepackage{graphicx}
\usepackage{hyperref}
\usepackage{makecell}

\newcommand{\settitolo}[1]{\newcommand{\titolo}{#1\\}}
\newcommand{\setprogetto}[1]{\newcommand{\progetto}{#1\\}}
\newcommand{\setcommittenti}[1]{\newcommand{\committenti}{#1\\}}
\newcommand{\setredattori}[1]{\newcommand{\redattori}{#1\\}}
\newcommand{\setrevisori}[1]{\newcommand{\revisori}{#1\\}}
\newcommand{\setresponsabili}[1]{\newcommand{\responsabili}{#1\\}}
\newcommand{\setversione}[1]{
	\ifdefined\versione\renewcommand{\versione}{#1\\}
	\else\newcommand{\versione}{#1\\}\fi
}
\newcommand{\setdestuso}[1]{\newcommand{\uso}{#1\\}}
\newcommand{\setdescrizione}[1]{\newcommand{\descrizione}{#1\\}}

\newcommand{\makefrontpage}{
	\begin{titlepage}
		\begin{center}

		\includegraphics[width=0.4\textwidth]{\templates/4ourSquared_logo}\\

		{\Large 4OURSQUARED}\\[6pt]
		\href{mailto://4oursquared.unipd@gmail.com}{4oursquared.unipd@gmail.com}\\
		
		\ifdefined\progetto
		\vspace{1cm}
		{\Large\progetto}
		{\large\committenti}
		\else\fi
		
		\vspace{1.5cm}
		{\LARGE\titolo}
		
		\vfill
		
		\begin{tabular}{r | l}
		\multicolumn{2}{c}{\textit{Informazioni}}\\
		\hline
		
		\ifdefined\redattori
			\textit{Redattori} &
			\makecell[l]{\redattori}\\
		\else\fi
		\ifdefined\revisori
			\textit{Revisori} &
			\makecell[l]{\revisori}\\
		\else\fi
		\ifdefined\responsabili
			\textit{Responsabili} &
			\makecell[l]{\responsabili}\\
		\else\fi
		
		\ifdefined\versione
			\textit{Versione} & \versione
		\else\fi
		
		\textit{Uso} & \uso
		
		\end{tabular}
		
		\vspace{2cm}
		
		\ifdefined\descrizione
		Descrizione
		\vspace{6pt}
		\hrule
		\descrizione
		\else\fi
		\end{center}
	\end{titlepage}
}
\usepackage{hyperref}
\usepackage{array}
\usepackage{tabularx}
\usepackage{adjustbox}

\newcounter{verscount}
\setcounter{verscount}{0}
\newcommand{\addversione}[5]{
	\ifdefined\setversione
		\setversione{#1}
	\else\fi
	\stepcounter{verscount}
	\expandafter\newcommand%
		\csname ver\theverscount \endcsname{#1&#2&#3&#4&#5}
}

\newcommand{\listversioni}{
	\ifnum\value{verscount}>1
		\csname ver\theverscount \endcsname
		\addtocounter{verscount}{-1}
		\\\hline
		\listversioni
	\else
		\csname ver\theverscount \endcsname\\\hline
	\fi
}

\newcommand{\makeversioni}{
	\begin{center}
		\begin{tabularx}{\textwidth}{|c|c|c|c|X|}
		\hline
		\textbf{Versione} & \textbf{Data} & \textbf{Redattore} & \textbf{Verificatore} & \textbf{Descrizione} \\
		\hline
		\listversioni
		\end{tabularx}
	\end{center}
	\clearpage
}

\settitolo{Norme di Progetto}
\setprogetto{Lumos Minima}
\setcommittenti{Imola Informatica}
\setredattori{Ceccato Francesco \\ Soldà Matteo}
\setdestuso{interno}
\setdescrizione{
Questo documento contiene le procedure, gli strumenti e i criteri di qualità che verranno usati nel progetto.
}

\addversione{0.0.0}{24/03/2023}{Soldà Matteo}{Brotto Romina}{Prima stesura.}
\addversione{0.0.1}{19/04/2023}{Soldà Matteo}{}{Terminata Sezione Documentazione}
\addversione{0.0.2}{20/04/2023}{Ceccato Francesco}{Soldà Matteo}{Terminata Sezione Convenzioni di Codifica}

\begin{document}
\makefrontpage \makeversioni
\tableofcontents
\newpage

\section{Introduzione}
\subsection{Scopo del Documento}
Il documento si prefigge lo scopo di definire i metodi e le attività chiave
legate al \textit{Way of Working}. Tutti i membri del gruppo si impegnano a
seguire quanto riportato, al fine di ottenere consistenza e coerenza in tutti i
documenti e in tutte le attività del gruppo, per raggiungere in modo efficace ed
efficiente la realizzazione del prodotto finale.

\subsubsection{Scopo del prodotto}
TODO: inserire quello che abbiamo previsto nella candidatura e quello di cui
abbiamo parlato nella presentazione.

\subsubsection{Glossario}
TODO: redigere un Glossario in un documento a parte oppure integrarlo qui;
Indicare termine nel glossario con testo \textit{corsivo} e con una 'G' in pedice.

\section{Processi Primari}
\subsection{Fornitura}
\subsubsection{Scopo}
Lo scopo del processo di fornitura è quello di descrivere e determinare ogni
compito ed attività svolta dal fornitore, al fine di comprendere e soddisfare le
richieste del proponente. 
La suddivisione e le tempistiche delle attività da svolgere saranno definite dal documento
\textit{Piano di Progetto}.

\subsubsection{Aspettative}
Le aspettative determinate dal processo di fornitura sono le seguenti:
\begin{itemize}
    \item Individuare e definire i bisogni che il prodotto deve soddisfare;
    \item Definire i requisiti e i vincoli dei processi;
    \item Stimare i costi per la realizzazione del prodotto;
    \item Ottenere dei feedback da proponente e committenti riguardanti il
    lavoro svolto;
    \item Chiarire eventuali dubbi sorti durante il progetto;
\end{itemize}

\subsubsection{Documenti}
\paragraph{\textit{Piano di Progetto}}
\\
Il \textit{Piano di Progetto} è il documento che viene redatto e mantenuto
durante tutta la durata del progetto. Al suo interno comprende:
\begin{itemize}
    \item \textbf{Analisi dei rischi:} vengono analizzati i rischi che possono
    essere individuati durante il corso del progetto e vengono descritte le
    strategie messe in atto al fine di contenerne la gravità;
    \item \textbf{Pianificazione del lavoro;}
    \item \textbf{Preventivo e consuntivo delle ore e dei costi.}
\end{itemize}
\paragraph{\textit{Piano di Qualifica}}
\\
Al fine di produrre materiale di qualità, viene redatto il documento
\textit{Piano di Qualifica}, che descrive le norme riguardanti la qualità dei prodotti e dei
processi del gruppo. Al suo interno comprende:
\begin{itemize}
    \item \textbf{Qualità di processo;}
    \item \textbf{Qualità di prodotto;}
    \item \textbf{Specifiche dei test;}
    \item \textbf{Resoconto delle attività di verifica.}
\end{itemize}

\subsubsection{Strumenti}
Nel processo di fornitura, il gruppo utilizza i seguenti strumenti per produrre
i documenti precedentemente elencati:
\begin{itemize}
    \item \textbf{Google Calendar:} sistema di calendari, usato per inserire le
    scadenze e gli eventi significativi riguardanti il gruppo;
    \item \textbf{Microsoft PowerPoint (TODO: Google Presentazioni?):} è un
    software facente parte del pacchetto suite Microsoft Office, permette la
    realizzazione delle presentazioni del gruppo.
    \item \textbf{TODO: programma per Gantt}
\end{itemize}

\subsubsection{Repository Pubblica}
Durante lo sviluppo del progetto, il codice e la documentazione saranno depositati in una repository pubblica. La parte di interesse del proponente e del committente riguarda la cartella \textit{public}. \\
La cartella sopra citata contiene la documentazione necessaria alle revisioni di periodo con il committente. Tale cartella rispecchia la struttura della cartella \textit{docs}, situata nella parte privata. \\
I file che prevedono versionamento, riporteranno al loro interno la numerazione relativa alla versione attuale e un riassunto delle versioni precedenti. \\ 
Ogni modifica della parte pubblica deve essere approvata, tramite \textit{pull request}, dal responsabile.

\subsubsection{Contatti}
Tutti i contatti ufficiali, sia con il proponente che con il committente, avverranno tramite la mail del gruppo \href{mailto:4oursquared.unipd@gmail.com}{4oursquared.unipd@gmail.com}.\\
Le mail per la presentazione ad una revisione di periodo necessita di una conferma unanime, valutata la preparazione. Questa mail deve contenere la lettera di presentazione con i link ai vari documenti aggiunti/aggiornati nella parte pubblica della repository.

\subsection{Sviluppo}
\subsubsection{Scopo}
Il processo di sviluppo ha come scopo quello di descrivere le attività e i task di
analisi, progettazione, codifica, test, installazione e accettazione riguardanti
il prodotto software in sviluppo.

\subsubsection{Aspettative}
Le aspettative prevedono:
\begin{itemize}
    \item Determinazione dei requisiti del prodotto;
    \item Determinazione dei vincoli tecnologici e di design;
    \item Determinazione degli obiettivi di sviluppo;
    \item Realizzazione del prodotto finale, superando tutti i test individuati
    e soddisfando i requisiti e le aspettative del proponente.
\end{itemize}

\subsubsection{\textit{Analisi dei Requisiti}}

\paragraph{Scopo}
Lo scopo dell'\textit{Analisi dei Requisiti} è quello di:
\begin{itemize}
    \item Identificare requisiti obbligatori ed auspicabili richiesti dal
    proponente, tramite lo studio del capitolato;
    \item Supportare l'attività di pianificazione, fornendo informazioni utili
    al computo della mole di lavoro;
    \item Supportare l'attività di verifica, facilitandone il tracciamento dei requisiti.
\end{itemize}

\paragraph{Aspettative}
L'attività si pone come risultato la creazione di un documento contenente tutti
i requisiti richiesti dal proponente, includendo anche il punto di vista
dell'utente che utilizzerà i prodotti. Sarà presente una sezione legata al
tracciamento dai requisiti ai casi d'uso e viceversa.

\paragraph{TBD: Casi d'Uso}
Un Caso d'Uso è l'insieme degli scenari che prevedono uno stesso obiettivo per
un utente. Vengono descritte le interazioni con il sistema da parte di uno o più
attori, senza fornire alcun dettaglio implementativo. È formato da:
\begin{itemize}
    \item \textbf{Diagramma \textit{UML}} (opzionale): indica la relazione con
    altri Casi d'Uso;
    \item \textit{\textbf{TBD: Intestazione}}, nel formato:
    \begin{center}
        \begin{BVerbatim}
UC [Codice] - [Numero Caso d'Uso].[Sottocaso] [Titolo]
        \end{BVerbatim}   
    \end{center}
    dove:
    \begin{itemize}
        \item \Verb^UC ^indica "\textit{Use Case}", ovvero "Caso d'Uso";
        \item \Verb^[Codice] ^è il codice identificativo del Caso d'Uso, 
        \item \Verb^[Titolo] ^è il titolo del Caso d'Uso. 
    \end{itemize}
    \item \textbf{\textit{Attore/i} primario/i};
    \item \textbf{\textit{Attore/i} secondario/i} (opzionale);
    \item \textbf{Descrizione};
    \item \textbf{Scenario principale};
    \item \textbf{Estensioni};
    \item \textbf{Inclusioni};
    \item \textbf{Precondizioni};
    \item \textbf{Postcondizioni};
    \item \textbf{Generalizzazioni}.
\end{itemize}

\paragraph{TBD: Requisiti}
I requisiti ricavati dall'analisi approfondita del capitolato, dalle discussioni
con i componenti del gruppo e dal confronto con il proponente, vengono suddivisi
in tre tipologie:
\begin{itemize}
    \item \textbf{Funzionali:} descrivono le funzionalità e il comportamento
    che il prodotto deve presentare;
    \item \textbf{Qualità:} descrivono i vincoli sulla qualità del prodotto e
    dei suoi componenti;
    \item \textbf{Vincolo:} descrivono i vincoli legati all'implementazione del
    prodotto, come ad esempio sulla tecnologia da utilizzare.
\end{itemize}
Ogni requisito ha una determinata importanza, di seguito descritta:
\begin{itemize}
    \item \textbf{Obbligatorio}
    \item \textbf{Desiderabile}
    \item \textbf{Opzionale}
\end{itemize}
Ogni requisito viene presentato nel formato univoco descritto di seguito: \textbf{TBD}
\begin{center}
    \begin{BVerbatim}
R[Tipologia][Caso d'Uso Relativo].[Sottocaso] - [Importanza]
    \end{BVerbatim}   
\end{center}
Dove:
\begin{itemize}
    \item \Verb^R^: Acronimo di "Requisito";
    \item \Verb^[Tipologia]^: indica la tipologia del requisito tra le seguenti:
    \begin{itemize}
        \item \textbf{F:} Funzionale;
        \item \textbf{Q:} Qualità;
        \item \textbf{V:} Vincolo;
    \end{itemize}
    \item \Verb^[Importanza]^: indica l'importanza del requisito tra le seguenti:
    \begin{itemize}
        \item \textbf{O:} Obbligatorio;
        \item \textbf{D:} Desiderabile;
        \item \textbf{P:} Opzionale.
    \end{itemize}
\end{itemize}

\subsubsection{Organizzazione dei File}
Tutta la documentazione sarà contenuta all'interno della cartella \textit{docs/}. Questa sarà inoltre suddivisa in \textit{interna} ed \textit{esterna}, le quali conterranno rispettivamente la documentazione interna al gruppo e quella da condividere con il proponente e con il committente. \\
I verbali interni saranno conservati nella sottocartella \textit{docs/interni/verbali} mentre quelli esterni saranno conservati in \textit{docs/esterni/verbali}.\\
Per ogni documento esisterà una nuova sottocartella omonima.

\subsubsection{Convenzioni di Codifica}
\textbf{Convenzioni Linguistiche} \newline
\begin{itemize}
    \item Ogni elemento del codice deve essere scritto in lingua inglese, ad eccezione dei commenti, che dovrebbero essere scritti in italiano per evitare incomprensioni.
\end{itemize}
\paragraph{}\\
\textbf{Convenzioni di Nomenclatura File}
\begin{itemize}
    \item Un file che contiene l'interfaccia o l'implementazione di una sola ed esclusiva classe (ad eccezione di classi interne e sottoclassi) dovrebbe essere individuato nel seguento modo: \texttt{NomeClasse.estensione}.
\end{itemize}
\paragraph{} \\
\textbf{Convenzioni Stilistiche} 
\begin{table}[ht]
\begin{tabular}{|| c || c | c | c | c |}
    \hline
        & Java & JavaScript & Python & TypeScript \\
    \hline \hline
    Funzioni e Metodi & camelCase() & camelCase() & snake\_case() & camelCase() \\
    \hline
    Classi & \multicolumn{4}{c|}{PascalCase} \\
    \hline
    Interfacce & PascalCase & PascalCase & NA & PascalCase \\
    \hline
    Namespace e Package & lowercase & NA & NA & camelCase \\
    \hline
    Costanti & \multicolumn{4}{c|}{SCREAMING\_SNAKE\_CASE} \\
    \hline  
    Variabili Locali e Attributi & camelCase & camelCase & snake\_case & snake\_case \\
    \hline
    Variabili Globali & NA & camelCase & snake\_case & snake\_case \\
    \hline
\end{tabular}
\end{table}
\vspace*{0.5cm} \newline
A seguire, in JavaScript, le funzioni e i metodi privati anteporranno il simbolo di underscore prima del nome.\newline
Per quanto riguarda invece i linguaggi di markup, i tag e gli attributi andranno scritti in lowercase se primitivi o in kebab-case se custom o complessi. \newline
Inoltre:
\begin{itemize}
    \item I membri di una classe dovrebbero essere dichiarati e/o definiti nel seguente ordine:
    \begin{itemize}
        \item Variabili di classe (statiche)
        \item Variabili di istanza (attributi) \\
        Nel seguente ordine (ove possibile):
        \begin{enumerate}
            \item public
            \item protected
            \item private
        \end{enumerate}
        \item Costruttore
        \item Distruttore (se richiesto)
        \item Metodi
    \end{itemize}
    \item Un blocco di codice delimitato da parentesi graffe dovrebbe essere strutturato apponendo le parentesi graffe ognuna in una nuova riga;
    \item Si predilige l'uso della tabulazione per indentare la porzione di codice inclusa in un blocco.
\paragraph{}\\
\end{itemize}
\textbf{CSS}
\begin{itemize}
    \item Mantenere un file chiamato \texttt{nomepagina.css} per la versione desktop di ciascuna pagina;
    \item Mantenere un file chiamato \texttt{global.css} per le regole che si applicano ad ogni aspetto del sito, in particolare ai moduli universali quali, a titolo esemplificativo e non esaustivo, header e footer;
    \item Mantenere un file chiamato \texttt{mini.css} con accorgimenti per la versione mobile
    \item Specificare le unità di misura in \textit{em}, \textit{rem}, \textit{\%}, \textit{vw}, \textit{vh} (unità relative), a parte per i valori delle seguenti proprietà, i quali possono essere espressi in \textit{px}:
    \begin{itemize}
        \item \textit{text-shadow}
        \item \textit{box-shadow}
    \end{itemize}
\end{itemize}
\section{Processi Secondari}
\subsection{Documentazione}
\subsubsection{Ciclo di Vita}
Ogni documento seguirà il seguente flusso:
\begin{itemize}
    \item \textbf{Pianificazione} \\ Il documento viene progettato ad alto livello secondo quanto richiesto.
    \item \textbf{Redazione} \\ Un membro del gruppo, nel ruolo di redattore, stila il documento.
    \item \textbf{Revisione} \\ Un membro del gruppo, nel ruolo di revisore, controlla l'assenza di errori grammaticali e il contenuto sia conforme alle norme di progetto.
    \item \textbf{Approvazione} \\ Un membro del gruppo, nel ruolo di responsabile, verifica che il contenuto del documento sia corretto.
\end{itemize}

\subsubsection{Strumenti Utilizzati}
Per la stesura dei documenti verrà utilizzato il linguaggio \LaTeX. \\
Affinché i documenti risultino stilisticamente coerenti, dovranno includere al loro interno il file \textit{docs/template/style.tex} per lo stile generale e il file \textit{docs/template/front\_page.tex} per utilizzare il template della prima pagina.

\subsubsection{Struttura}
Ogni documento sarà così strutturato:
\begin{enumerate}
    \item Pagina di intestazione
    \item Pagina con lista delle versioni, ove necessario
    \item Pagina indice, ove necessario
    \item Una o più pagine di contenuto
\end{enumerate}

Le varie sezioni saranno così definite: \\

\textbf{Pagina di Intestazione} 
\begin{itemize}
    \item Logo e nome del gruppo
    \item Nome del progetto e azienda cliente
    \item Mail di contatto
    \item Titolo del documento
    \item Info generali
    \begin{itemize}
        \item Redattori
        \item Revisori
        \item Responsabile
        \item Versione, ove necessario
        \item Destinazione d'uso (interno o esterno)
    \end{itemize}
    \item Sinossi
\end{itemize}

\textbf{Elenco delle Versioni} \\
L'elenco delle versioni è una tabella così definita:
\begin{itemize}
    \item Versione
    \item Data
    \item Redattore
    \item Revisore
    \item Descrizione
\end{itemize}
Le righe della tabella sono organizzate in ordine cronologico inverso. 
\newline \newline
\textbf{Indice} \\
L'indice è stato creato utilizzando il comando \textit{\textbackslash tableofcontents}
\newline \newline
\textbf{Norme di Progetto}
\begin{itemize}
    \item \textbf{Scopo:} Lo scopo del documento è quello di definire le procedure, gli strumenti e criteri di qualità al fine di stabilire un \textit{Way of Working}. Ogni membro del gruppo sarà tenuto a rispettare le indicazioni del documento presentate
    \item \textbf{Titolo:} Norme di Progetto
    \item \textbf{Nome del File:} $norme\_di\_progetto.tex$
\end{itemize} 
\paragraph{}\\
\textbf{Verbali}
\begin{itemize}
    \item \textbf{Scopo:} Lo scopo dei verbali è quello di rendicontare ciò che viene discusso durante una riunione, sia interna che esterna, delineando gli impegni che ne derivano
    \item \textbf{Titolo:} Ogni verbale sarà titolato con "$Verbale \ del \ <data> \ [con <esterni>]$"
    \item \textbf{Nome del File:} Tutti i file saranno chiamati $<yyyy>\_<mm>\_<dd>\_<tipo>.tex$ dove $<yyyy>\_<mm>\_<dd>$ indica la data nella quale si è effettuatal a riunione nel formato anno-mese-giorno; <tipo> indica se la riunione è avvenuta tra i soli membri del gruppo "I" o con persone esterne "E".
    \item \textbf{Inidice:} Non presente in quanto non utile
    \item \textbf{Struttura:}
    \begin{itemize}
        \item Data, ora e durata;
        \item Luogo;
        \item Partecipanti (interni ed esterni, raggruppati per appartenenza e ordinati alfabeticamente secondo il cognome);
        \item Ordine del giorno;
        \item Conclusioni derivanti dalla riunione;
        \item Impegni assunti.
    \end{itemize}
    \item \textbf{Stesura:} All'inizio di ogni riunione verrà nominato un membro del gruppo che redigerà il verbale e un membro che lo validerà. Chi redige il verbale avrà 24 ore per completare il lavoro, lo stesso tempo sarà concesso al validatore per approvarlo e segnalare le modifiche da applicare.
\end{itemize}
\paragraph{}\\
\textbf{Piano di Progetto}
\begin{itemize}
    \item \textbf{Scopo:} Lo scopo del documento è quello di definire gli obiettivi da raggiungere, stabilendo le tempistiche e il responsabile, tenendo traccia dei progeressi e valutando i coste sostenuti rispetto a quelli preventivati.
    \item \textbf{Titolo:} Piano di Progetto
    \item \textbf{Nome del File:} 
\end{itemize}
\paragraph{}\\
\textbf{Piano di Qualifica}
\begin{itemize}
    \item \textbf{Scopo:} Lo scopo del documento è quello di definire le metriche e i requisiti minimi di qualità affinchè un prodotto del progetto possa essere approvato.
    \item \textbf{Titolo:} Piano di Qualifica
    \item \textbf{Nome del File:} 
\end{itemize}
\subsubsection{Convenzioni Documentali}
\begin{itemize}
    \item \textbf{Riferimento a file o cartelle:}
    \begin{itemize}
        \item Per fare riferimento al nome di un file o di una cartella si utilizza il \texttt{testo monospaziato};
        \item Il nome di una cartella deve sempre terminare con \texttt{/}.
    \end{itemize}
    \item \textbf{Stringhe e nomi:}
    \begin{itemize}
        \item Le stringhe vanno scritte tra doppi apici;
        \item Per indicare un parametro rappresentate una stringa si usa il testo tra parentesi angolari. Tale parametro non può contenere spazi;
        \item Per indicare una parte di nome o stringa opzionale si usa il testo racchiuso tra parentesi quadre.
    \end{itemize}
    \item \textbf{Riferimento tra documenti:}
    \begin{itemize}
        \item Riferimenti interni al documento: comando \texttt{$\backslash$ref};
        \item Riferimenti esterni al documento: nome della sezione del documento a cui si fa riferimento in corsivo.
    \end{itemize}
\end{itemize}
\subsection{Versionamento}
\subsubsection{Repository Github}
Per il versionamento del progetto si è deciso di utilizzare \textit{Git} sulla piattaforma \textit{GitHub}. La repository è raggiungibile tramite l'indirizzo \href{https://github.com/4ourSquared/LumosMinima}{https://github.com/4ourSquared/LumosMinima}.
\subsubsection{Convenzioni di Versionamento}
Per il versionamento si farà uso del Versionamento Semantico, secondo le \href{https://semver.org/lang/it/#specifica-di-versionamento-semantico-semver}{linee guida}.\newline
Questo tipo di versionamento si fonda sulle seguenti caratteristiche principali:
\begin{enumerate}
    \item Un numero di versione deve essere nella forma $X.Y.Z$ dove $X,Y,Z$ sono numeri interi non negativi e non devono contenere zeri iniziali;
    \item La versione \textit{major zero} (es: $0.X.Y$) serve solo per lo sviluppo iniziale e indica una versione non stabile del prodotto;
    \item La versione \textit{patch Z} (es: $x.y.Z$) deve essere incrementata solo se sono state introdotte correzioni retrocompatibili del testo (modifica da parte del reddatore);
    \item La versione \textit{minor Y} (es: $x.Y.z$) deve essere incrementata se nella documentazione è stata introdotta una nuova modifica retrocompatibile (approvazione da parte del verificatore);
    \item la versione \textit{major X} (es: $X.y.z$) deve essere incrementata solo se nella documentazione è stata introdotta una modifica non retrocompatibile con le versioni precedenti (pubblicazione da parte del responsabile).
\end{enumerate}
\subsubsection{Comandi Utili}
\begin{itemize}
    \item \textbf{Sincronizzazione con la repository remota:} \texttt{git pull};
    \item \textbf{Creazione di un nuovo branch locale:} \texttt{git branch <nome\_branch>};
    \item \textbf{Passaggio ad un branch locale esistente:} \texttt{git checkout <nome\_branch>};
    \item \textbf{Aggiunta delle modifiche alla stage area:} \texttt{git add .};
    \item \textbf{Creazione del commit con le modifiche:} \texttt{git commit -m <descrizione>};
    \item \textbf{Push sul remote in un branch remoto non esistente:} \texttt{git push --set-upstream origin <nome\_branch>};
    \item \textbf{Push sul branch remoto esistente omonimo del branch locale:} \texttt{git push}.
\end{itemize}
Escludendo per il momento l'utilizzo di \textit{git flow}, questi risultano i comandi da utilizzare.
\end{document}
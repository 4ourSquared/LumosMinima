\documentclass[a4paper, 12pt]{article}

\newcommand{\templates} {../../../template}
\usepackage[a4paper, margin=2.5cm]{geometry}
\usepackage{fancyvrb}
\usepackage{stackengine}
\usepackage{lastpage}

\usepackage{enumitem}
\setlist[itemize]{noitemsep}
\setlist[enumerate]{noitemsep}

\let\oldpar\paragraph
\renewcommand{\paragraph}[1]{\oldpar{#1\\}\noindent}

% Avoid dots in the table of contents, it mess with the gulpease calculation
\makeatletter
\renewcommand{\@dotsep}{10000} 
\makeatother

\newcommand{\makeindexdetails}{
	\pagestyle{fancy}
	\lhead{4ourSquared} \chead{\includegraphics[width=1cm]{../../template/4ourSquared_logo} } \rhead{Versione e Indice}
	\pagenumbering{Roman}
}

\newcommand{\makecontentsdetails}[1]{
	\clearpage
    \renewcommand{\footrulewidth}{0.4pt}
	\pagestyle{fancy}
	\lhead{4ourSquared} \chead{\includegraphics[width=1cm]{../../template/4ourSquared_logo}} \rhead{\nouppercase{\leftmark}}
	\pagenumbering{arabic}
    \lfoot{#1} \cfoot{} \rfoot{\thepage / \pageref{LastPage}}
}
\usepackage{graphicx}
\usepackage{hyperref}
\usepackage{makecell}

\newcommand{\settitolo}[1]{\newcommand{\titolo}{#1\\}}
\newcommand{\setprogetto}[1]{\newcommand{\progetto}{#1\\}}
\newcommand{\setcommittenti}[1]{\newcommand{\committenti}{#1\\}}
\newcommand{\setredattori}[1]{\newcommand{\redattori}{#1\\}}
\newcommand{\setrevisori}[1]{\newcommand{\revisori}{#1\\}}
\newcommand{\setresponsabili}[1]{\newcommand{\responsabili}{#1\\}}
\newcommand{\setversione}[1]{
	\ifdefined\versione\renewcommand{\versione}{#1\\}
	\else\newcommand{\versione}{#1\\}\fi
}
\newcommand{\setdestuso}[1]{\newcommand{\uso}{#1\\}}
\newcommand{\setdescrizione}[1]{\newcommand{\descrizione}{#1\\}}

\newcommand{\makefrontpage}{
	\begin{titlepage}
		\begin{center}

		\includegraphics[width=0.4\textwidth]{\templates/4ourSquared_logo}\\

		{\Large 4OURSQUARED}\\[6pt]
		\href{mailto://4oursquared.unipd@gmail.com}{4oursquared.unipd@gmail.com}\\
		
		\ifdefined\progetto
		\vspace{1cm}
		{\Large\progetto}
		{\large\committenti}
		\else\fi
		
		\vspace{1.5cm}
		{\LARGE\titolo}
		
		\vfill
		
		\begin{tabular}{r | l}
		\multicolumn{2}{c}{\textit{Informazioni}}\\
		\hline
		
		\ifdefined\redattori
			\textit{Redattori} &
			\makecell[l]{\redattori}\\
		\else\fi
		\ifdefined\revisori
			\textit{Revisori} &
			\makecell[l]{\revisori}\\
		\else\fi
		\ifdefined\responsabili
			\textit{Responsabili} &
			\makecell[l]{\responsabili}\\
		\else\fi
		
		\ifdefined\versione
			\textit{Versione} & \versione
		\else\fi
		
		\textit{Uso} & \uso
		
		\end{tabular}
		
		\vspace{2cm}
		
		\ifdefined\descrizione
		Descrizione
		\vspace{6pt}
		\hrule
		\descrizione
		\else\fi
		\end{center}
	\end{titlepage}
}
\usepackage{hyperref}
\usepackage{array}
\usepackage{tabularx}
\usepackage{adjustbox}

\newcounter{verscount}
\setcounter{verscount}{0}
\newcommand{\addversione}[5]{
	\ifdefined\setversione
		\setversione{#1}
	\else\fi
	\stepcounter{verscount}
	\expandafter\newcommand%
		\csname ver\theverscount \endcsname{#1&#2&#3&#4&#5}
}

\newcommand{\listversioni}{
	\ifnum\value{verscount}>1
		\csname ver\theverscount \endcsname
		\addtocounter{verscount}{-1}
		\\\hline
		\listversioni
	\else
		\csname ver\theverscount \endcsname\\\hline
	\fi
}

\newcommand{\makeversioni}{
	\begin{center}
		\begin{tabularx}{\textwidth}{|c|c|c|c|X|}
		\hline
		\textbf{Versione} & \textbf{Data} & \textbf{Redattore} & \textbf{Verificatore} & \textbf{Descrizione} \\
		\hline
		\listversioni
		\end{tabularx}
	\end{center}
	\clearpage
}

\settitolo{Verbale del 21/08/2023}
\setredattori{Soldà Matteo}
\setrevisori{}
\setdestuso{interno}
\setdescrizione{
    Verbale dell'incontro del 21/08/2023
}

\begin{document}

\makefrontpage

\section*{Informazioni Generali}
\begin{itemize}
    \item Data e Ora: 21/08/2023 15:00
    \item Durata: 1h
    \item Piattaforma: Discord
    \item Partecipanti
    \begin{itemize}
        \item Alberti Nicolas
        \item Brotto Romina
        \item Cavaliere Erica
        \item Ceccato Francesco
        \item Salami Lorenzo
        \item Soldà Matteo
    \end{itemize}
\end{itemize}

\section*{O.D.G}
\begin{itemize}
    \item Discussione del Lavoro Svolto
    \item Definizione Prossimi Obiettivi
    \item Varie ed Eventuali.
\end{itemize}

\section*{Conclusioni}
Durante la riunione è stato discusso quanto svolto durante la settimana precedente, cercando di proporre alternative migliorative in vista della stesura dei documenti essenziali per la fase di \textit{Product Baseline - PB} prevista per la prima metà di settembre.\\\\
Durante la settimana sono stati implementate buona parte delle modali di conferma, sono stati implementati i primi unit test, è stata predisposta e applicata la modifica del sistema della webapp derivante dal \textit{Verbale Esterno del 2023/08/16} effettuato con il Sig. Patera.\\\\
Inoltre, è stata implementata la prima versione del meccanismo che permette di collegare i sensori e i lampioni di una determinata area, permettendo l'incremento e la successiva riduzione dell'intensità luminosa di un lampione qualora il sensore rilevasse un movimento, sia che i lampioni siano impostati in modalità \textit{pull} che in modalità \textit{push}. \\\\
Infine, è in fase di sviluppo un sistema di login/logout basato su \textit{JWT - JSON Web Token\textsubscript{G}} che permetta ad un utente di visualizzare solo le pagine che siano di propria competenza. 
\section*{Impegni Assunti}

Per la documentazione nella repository LumosMinima:
\begin{itemize}
    \item Erica si occupa di spostare i documenti nelle cartelle corrette, quali verbali e lettera di presentazione (issue \#92);
    \item Romina si occupa di aggiornare il \textit{Piano di Progetto} (issue \#97);
    \item Lorenzo provvederà ad aggiornare e modificare il \textit{Piano di Qualifica} (issue \#96);
    \item Nicolas provvederà a redigere una versione di base della Specifica Architetturale e della Specifica Tecnica (issues \#106 - \#107).
\end{itemize}

Per il codice presente nella repository LumosMinima-code:
\begin{itemize}
    \item Matteo si occuperà dell'implementazione della modifica della luminosità temporizzata per le aree illuminate (issue \#11);
    \item Erica e Francesco procederanno con gli unit test (issue \#1);
    \item Matteo e Francesco di occuperanno del meccanismo di login (issue \#5).
\end{itemize}

\end{document}

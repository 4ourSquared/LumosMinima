\documentclass[a4paper, 12pt]{article}

\newcommand{\templates} {../../../template}
\usepackage[a4paper, margin=2.5cm]{geometry}
\usepackage{fancyvrb}
\usepackage{stackengine}
\usepackage{lastpage}

\usepackage{enumitem}
\setlist[itemize]{noitemsep}
\setlist[enumerate]{noitemsep}

\let\oldpar\paragraph
\renewcommand{\paragraph}[1]{\oldpar{#1\\}\noindent}

% Avoid dots in the table of contents, it mess with the gulpease calculation
\makeatletter
\renewcommand{\@dotsep}{10000} 
\makeatother

\newcommand{\makeindexdetails}{
	\pagestyle{fancy}
	\lhead{4ourSquared} \chead{\includegraphics[width=1cm]{../../template/4ourSquared_logo} } \rhead{Versione e Indice}
	\pagenumbering{Roman}
}

\newcommand{\makecontentsdetails}[1]{
	\clearpage
    \renewcommand{\footrulewidth}{0.4pt}
	\pagestyle{fancy}
	\lhead{4ourSquared} \chead{\includegraphics[width=1cm]{../../template/4ourSquared_logo}} \rhead{\nouppercase{\leftmark}}
	\pagenumbering{arabic}
    \lfoot{#1} \cfoot{} \rfoot{\thepage / \pageref{LastPage}}
}
\usepackage{graphicx}
\usepackage{hyperref}
\usepackage{makecell}

\newcommand{\settitolo}[1]{\newcommand{\titolo}{#1\\}}
\newcommand{\setprogetto}[1]{\newcommand{\progetto}{#1\\}}
\newcommand{\setcommittenti}[1]{\newcommand{\committenti}{#1\\}}
\newcommand{\setredattori}[1]{\newcommand{\redattori}{#1\\}}
\newcommand{\setrevisori}[1]{\newcommand{\revisori}{#1\\}}
\newcommand{\setresponsabili}[1]{\newcommand{\responsabili}{#1\\}}
\newcommand{\setversione}[1]{
	\ifdefined\versione\renewcommand{\versione}{#1\\}
	\else\newcommand{\versione}{#1\\}\fi
}
\newcommand{\setdestuso}[1]{\newcommand{\uso}{#1\\}}
\newcommand{\setdescrizione}[1]{\newcommand{\descrizione}{#1\\}}

\newcommand{\makefrontpage}{
	\begin{titlepage}
		\begin{center}

		\includegraphics[width=0.4\textwidth]{\templates/4ourSquared_logo}\\

		{\Large 4OURSQUARED}\\[6pt]
		\href{mailto://4oursquared.unipd@gmail.com}{4oursquared.unipd@gmail.com}\\
		
		\ifdefined\progetto
		\vspace{1cm}
		{\Large\progetto}
		{\large\committenti}
		\else\fi
		
		\vspace{1.5cm}
		{\LARGE\titolo}
		
		\vfill
		
		\begin{tabular}{r | l}
		\multicolumn{2}{c}{\textit{Informazioni}}\\
		\hline
		
		\ifdefined\redattori
			\textit{Redattori} &
			\makecell[l]{\redattori}\\
		\else\fi
		\ifdefined\revisori
			\textit{Revisori} &
			\makecell[l]{\revisori}\\
		\else\fi
		\ifdefined\responsabili
			\textit{Responsabili} &
			\makecell[l]{\responsabili}\\
		\else\fi
		
		\ifdefined\versione
			\textit{Versione} & \versione
		\else\fi
		
		\textit{Uso} & \uso
		
		\end{tabular}
		
		\vspace{2cm}
		
		\ifdefined\descrizione
		Descrizione
		\vspace{6pt}
		\hrule
		\descrizione
		\else\fi
		\end{center}
	\end{titlepage}
}
\usepackage{hyperref}
\usepackage{array}
\usepackage{tabularx}
\usepackage{adjustbox}

\newcounter{verscount}
\setcounter{verscount}{0}
\newcommand{\addversione}[5]{
	\ifdefined\setversione
		\setversione{#1}
	\else\fi
	\stepcounter{verscount}
	\expandafter\newcommand%
		\csname ver\theverscount \endcsname{#1&#2&#3&#4&#5}
}

\newcommand{\listversioni}{
	\ifnum\value{verscount}>1
		\csname ver\theverscount \endcsname
		\addtocounter{verscount}{-1}
		\\\hline
		\listversioni
	\else
		\csname ver\theverscount \endcsname\\\hline
	\fi
}

\newcommand{\makeversioni}{
	\begin{center}
		\begin{tabularx}{\textwidth}{|c|c|c|c|X|}
		\hline
		\textbf{Versione} & \textbf{Data} & \textbf{Redattore} & \textbf{Verificatore} & \textbf{Descrizione} \\
		\hline
		\listversioni
		\end{tabularx}
	\end{center}
	\clearpage
}

\settitolo{Verbale del 02/05/2023}
\setredattori{Brotto Romina}
\setrevisori{Alberti Nicolas}
\setdestuso{interno}
\setdescrizione{
    Verbale dell'incontro del 02/05/2023
}

\begin{document}

\makefrontpage

\section*{Informazioni Generali}
\begin{itemize}
    \item Data e Ora: 02/05/2023 10:00
    \item Durata: 1h30'
    \item Piattaforma: Discord
    \item Partecipanti
    \begin{itemize}
        \item Alberti Nicolas
        \item Brotto Romina
        \item Cavaliere Erica
        \item Ceccato Francesco
        \item Salami Lorenzo
        \item Soldà Matteo
    \end{itemize}
\end{itemize}

\section*{O.D.G}
\begin{itemize}
    \item Redazione Template per le Presentazioni del Diario di Bordo
    \item Definizione criteri di rotazione dei ruoli
    \item Prova simulatore
    \item Board di Progetto
\end{itemize}

\section*{Conclusioni}
\indent Il gruppo si è riunito Martedì 02 Maggio, anziche di Lunedì come previsto dalle Norme di Progetto, in via del tutto eccezionale a causa della festività del 1 Maggio.\newline
\indent Durante la riunione sono stati definiti i criteri di rotazione dei ruoli per ogni membro: tale rotazione verrà tarata sulla base della disponibilità del singolo membro del gruppo, cercando di mantenere equità tra i componenti. A tale scopo è stato introdotto un documento condiviso redatto con Google Fogli per il tracciamento delle ore.
\newline
\'E stato eseguito un test di prova del simulatore di lampadina fornito dall'Azienda.

\indent \'E inoltre stata aperta la Board di Progetto per la gestione delle issues. 
\newline
\'E stata prevista la modifica dei seguenti documenti, in accordo con la Board di Progetto presente su GitHub Projects:
\begin{itemize}
    \item Norme di Progetto: specifica della gestione e dell'uso della Project Board e delle issues.
\end{itemize}





\end{document}